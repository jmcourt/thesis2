\chapter{PANTHEON suite}

\label{app:PAN}

\par Below is the code for the entire suite of \textit{PANTHEON} (Python ANalytical Tools for High-energy Event data manipulatiON) that I used during the work presented in this thesis.  This code is also available at \url{https://github.com/jmcourt/PANTHEON}.  PANTHEON makes use of the Astropy \citep{Astropy}, Matplotlib \citep{Hunter_MatPlotLib}, Numpy, Scipy \citep{NumPy} and Numba \citep{Numba}.

\section{FITS Genie}

\par \textit{FITS Genie} is a script that allows the user to extract data from raw \texttt{FITS} files.  The script was designed to interface with \textit{RXTE} data, but there is also limited implementation with \textit{Suzaku}.  The script produces \texttt{.plotd} and \texttt{.speca} files, which can be further processed with \textit{Plot Demon} and \textit{Spec Angel}.

\begin{minted}[fontsize=\scriptsize]{python}
#! /usr/bin/env python

# |----------------------------------------------------------------------|
# |-----------------------------FITS GENIE-------------------------------|
# |----------------------------------------------------------------------|

# Call as ./fitsgenie.py FILE1 PROD_REQ [LCHAN] [HCHAN] [BINNING] [FOURIER RES] 
#   [FOURIER SEP] [BGEST] [FLAVOUR]
#
# Takes 1 FITS Event file and produces .speca and .plotd formatted products to be
#   analysed by plotdemon
# and specangel.
#
# Arguments:
#
#  FILE1
#   The absolute path to the file to be used.
#
#  PROD_REQ
#   The products requested by the user.  The following inputs are valid:
#      'spec','speca','s' will cause FITSGenie to produce only a .speca file as output
#      'plot','plotd','p' will cause FITSGenie to produce only a .plotd file as output
#      'both','all','b','a','sp','ps' will cause both files to be output
#
#  [LCHAN]
#   Optional: The lowest channel on the PCA instrument on RXTE which will be used to
#     populate the data.  Default of 0 (minimum).
#
#  [HCHAN]
#   Optional: The highest channel on the PCA instrument on RXTE which will be used to
#     populate the data.  Default of 255 (maximum).
#
#  [BINNING]
#   Optional: The size, in seconds, of bins into which data will be sorted.  Takes the
#     value of the time resolution of the data if not specified by the user.  Default
#     of 2^-15s
#
#  [FOURIER RES]
#   Optional: The size of the individual time windows in which the data is to be split.
#   Fourier spectra will be made of each of these windows.  Default of 128s.
#
#  [FOURIER SEP]
#   Optional: The separation of the startpoints of individual time windows in which the
#     data is to be split.  Fourier spectra will be made of each of these windows.
#     Default of 128s.
#
#  [BGEST]
#   Optional: The approximate average background count rate during the observation in
#     cts/s.  Default of 30cts/s.
#
#  [FLAVOUR]
#   Optional: A useful bit of text to put on plots to help identify them later on.

#-----User-set Parameters--------------------------------------------------------------

ptdbinfac=1                 # To save space and time, the time bins for saved 
                            #   plotdemon data will be greater than the time bins for
                            #   the not-saved specangel data by this factor.  Must be
                            #   power of 2.
spcbinfac=4096              # The binning factor for SpecAngel data to use when
                            #   searching for data peaks and troughs
usrmin=-13                  # The smallest time resolution to consider is 2^usrmin
                            #   seconds
version=6.1

#-----Welcoming Header-----------------------------------------------------------------

print ''
print '-------Running FITSGenie: J.M.Court, 2015-------'
print ''

#-----Importing Modules----------------------------------------------------------------

try:
   import sys
   import pan_lib as pan
   from astropy.io import fits
except ImportError:
   print 'Modules missing!  Aborting!'
   print ''
   print '------------------------------------------------'
   print ''
   exit()

#-----Checking Validity of Filename----------------------------------------------------

args=sys.argv
pan.argcheck(args,1)                    # Must give at least 1 args (the function call)
if len(args)<2:
   filename=raw_input('Filename: ')
   print ''
else:
   filename=args[1]                     # Fetch file name from arguments
if len(args)<3:
   print ''
   print 'FitsGenie can produce [P]lotDemon files, [S]pecangel files or [B]oth.'
   print ''
   prod_req=raw_input('Select product(s): ')
   print ''
else:
   prod_req=args[2]                     # Fetch products request from arguments

#-----Identifying Products-------------------------------------------------------------

if prod_req.lower() in ['spec','speca','s','both','all','a','b','ps','sp']:
   spec_on=True
   print '.speca File will be created!'
else:
   spec_on=False
   print '.speca File will NOT be created!'
if prod_req.lower() in ['plot','plotd','p','both','all','a','b','ps','sp']:
   plot_on=True
   print '.plotd File will be created!'
else:
   plot_on=False
   print '.plotd File will NOT be created!'
del prod_req
print ''

#-----Opening FITS file, identifying mission-------------------------------------------

try:
   assert filename[-6:] not in ('.speca','.plotd') # Don't try to open plotd/speca
                                                   #   files please...
   event=fits.open(filename)                       # Unleash the beast! [open the file]
except:
   print 'Could not open file "'+filename+'"!'
   print 'Aborting!'
   pan.signoff()
   exit()

from math import log                               # Import remaining modules.  This is
                                                   #   a slight speedup when running a
                                                   #   a script to attempt to FITSgenie
                                                   #   some valid files and some 
                                                   #   invalid files
from numpy import arange, array, histogram, zeros
from numpy import sum as npsum
from scipy.fftpack import fft
import pylab as pl

try:
   mission=event[1].header['TELESCOP']             # Fetch the name of the telescope
except:
   print 'Could not identify mission!'
   print 'Aborting!'
   pan.signoff()
   exit()
if mission in ['XTE','SUZAKU','SWIFT']:
   print mission,'data detected!'
else:
   print mission,'data not yet supported!'
   pan.signoff()
   exit()
if mission == 'XTE' :
   etype='channel'                                 # XTE requires an input of channel
                                                   #   IDs
   escale=''
   escaleb=''
   #try:
   import xtepan_lib as inst                       # Import XTE extraction functions
   #except:
   #   print 'XTE PANTHEON Library not found!  Aborting!'
   #   pan.signoff()
   #   exit()
elif mission == 'SUZAKU':
   etype='energy'                                  # SUZAKU requires an input of raw
                                                   #   energies
   escale='eV'
   escaleb=' (eV)'
   try:
      import szkpan_lib as inst                    # Import SUZAKU extraction functions
   except:
      print 'Suzaku PANTHEON Library not found!  Aborting!'
      pan.signoff()
      exit()
elif mission == 'SWIFT':
   etype='energy'                                  # SUZAKU requires an input of raw
                                                   #   energies
   escale='eV'
   escaleb=' (eV)'
   try:
      import swfpan_lib as inst                    # Import SUZAKU extraction functions
   except:
      print 'Swift PANTHEON Library not found!  Aborting!'
      pan.signoff()
      exit()
else:
   print "This error shouldn't happen...  sorry 'bout that!"
   pan.signoff()
   exit()
try:
   obsdata=inst.getobs(event,event[1].header['DATAMODE'],filename)       # Fetch object
   print event[1].header['DATAMODE'],'format detected.'
except:
   print 'Could not identify DATAMODE!'
   print 'Aborting!'
   pan.signoff()
   exit()
if event[1].header['DATAMODE'][:2] in ['B_','SB']:
   spec_on=False
   bin_dat=True
   print 'No .speca file can be produced!'
   if not plot_on:
      print 'Aborting!'
      pan.signoff()
      exit()
else:
   bin_dat=False

#-----Checking validity of remaining inputs--------------------------------------------

print 'Object =',obsdata[0]
print 'Obs_ID =',obsdata[1]
print ''
maxen=inst.maxen(event[1].header['DATAMODE'])            # Get the value of the
                                                         #   highest energy or channel
                                                         #   for the instrument
print 'Inputs:'
print ''
if len(args)>3:
   lowc=int(args[3])                                     # Collect minimum channel
                                                         #   label from user
   print 'Min Channel =',lowc
else:
   try:
      lowc=int(raw_input("Minimum "+etype+escaleb+": "))
   except:
      lowc=0
      print "Using min "+etype+" of 0"+escale+"!"
if len(args)>4:
   highc=int(args[4])                                    # Collect maximum channel
                                                         #   label from user
   print 'Max Channel =',highc
else:
   try:
      highc=int(raw_input("Maximum "+etype+escaleb+": "))
   except:
      highc=maxen
      print "Using max "+etype+" of "+str(maxen)+escale+"!"
if lowc<0:    lowc=0                                     # Force channels to be in
                                                         #   range 0,255
if highc>maxen: highc=maxen
if lowc>highc:
   print 'Invalid '+etype+'!  Aborting!'                 # Abort if user gives
                                                         #   lowc>highc
   pan.signoff()
   exit()
cs=str(int(lowc))+'-'+str(int(highc))
if len(args)>5:
   bszt=float(args[5])                                   # Collect binsize from inputs
                                                         #   if given, else ask user,
                                                         #   else use resolution
                                                         #   encoded in .fits file
   print 'Bin size (s)=',bszt
else:
   try:
      bszt=float(raw_input("Photon count bin-size (s): "))
   except:
      bszt=0
      print "Using max time resolution..."
if len(args)>6:
   foures=float(args[6])                                 # Collect Fourier resolution
                                                         #   from inputs if given, else
                                                         #   ask user, else use 128s
   print 'Fourier Res.=',foures
elif not spec_on:
   foures=16
else:
   try:
      foures=float(raw_input("Length of time per Fourier Window (s): "))
   except:
      foures=128
      print "Using 128s per spectrum..."
if len(args)>7:
   slide=float(args[7])                                  # Collect Fourier resolution
                                                         #   from inputs if given, else
                                                         #   ask user, else use 128s
   print 'Fourier Sep.=',slide
elif not spec_on:
   slide=16
else:
   try:
      slide=float(raw_input("Separation of Fourier Windows (s): "))
   except:
      slide=foures
      print "Using "+str(slide)+"s per spectrum..."
if len(args)>8:
   bgest=float(args[8])                                  # Collect background estimate
                                                         #   from inputs if given, else
                                                         #   ask user, else use 30c/s
   print 'Background  =',bgest
elif not spec_on:
   bgest=0
else:   
   try:
      bgest=float(raw_input("Estimate of background (c/s): "))
   except:
      bgest=30
      print "Using 30c/s background..."
   print ''
if len(args)>9:
   flavour=args[9]                                       # Collect flavour if given,
                                                         #   else flavourless
   print 'Flavour     =',flavour
else:
   flavour=''
print ''
wtype='Boxcar'                                           # Setting all windows to
                                                         #   BoxCar; will make 
                                                         #   transition to SpecAngel
                                                         #   4.0 smoother if this
                                                         #   takes a value

#-----Masking data---------------------------------------------------------------------

gti=inst.getgti(event)                                   # Extract GTI indices
datas=inst.getdat(event)                                 # Extract event data
print 'Discarding photons outside of '+etype+' range '\
   +str(lowc)+escale+'-'+str(highc)+escale+'...'
datas=inst.discnev(datas,event[1].header['DATAMODE'])    # Discarding non-events /
                                                         #   reformatting XTE Binned
                                                         #   data into a less awful
                                                         #   structure
if event[1].header['DATAMODE'][:2] == 'B_':
   olen=str(npsum(array(datas)))
else:
   olen=str(len(datas))
datas=inst.chrange(datas,lowc,highc,event[1].header)
tstart=inst.getini(event)
if event[1].header['DATAMODE'][:2] == 'B_':
   phcts=npsum(datas)
else:
   phcts=len(datas)
if float(olen)==0:
   print 'No photons!  Aborting!'
   pan.signoff()
   exit()
pcg=str(int(100*phcts/float(olen)))+'\%'
print str(phcts)+'/'+olen+' photons fall within '+etype+' range ('+pcg+')!'
if phcts==0:
   print 'Aborting!'
   pan.signoff()
   exit()
print ''

#-----Fetching Bin Size----------------------------------------------------------------

bsz=inst.getbin(event,event[1].header['DATAMODE']) # Fetch 'Binning' as the time
                                                   #   resolution of the data
ores=bsz
if bszt>bsz:                                       # If user enters a lower binning
                                                   #   resolution than maximum, use
                                                   #   that instead
   bsz=bszt
n=usrmin                                           # Rounding bsz to the nearest
                                                   #   (greater) power of 2
while (2**n)<bsz:
   n+=1
bsz=2**n
bsz=float(bsz)

#-----Fetching Time Axis---------------------------------------------------------------

print 'SpecAngel binsize rounded to 2^'+str(n)+'s ('+str(bsz)+'s)!'
print 'PlotDemon binsize rounded to 2^'+str(n+int(log(ptdbinfac,2)))\
   +'s ('+str(bsz*ptdbinfac)+'s)!'
times,datas=inst.gettim(datas,event[1].data,tstart,ores,event[1].header['DATAMODE'])
                     # ^ Extracting list of photon incident times as a separate object
pcwrds=inst.getwrd(datas,event[1].header['DATAMODE'])
sttim=times[0]
times=times-sttim

#-----Fetching Fourier Range Size------------------------------------------------------

if foures>max(times):
   foures=128
if slide>max(times):
   slide=foures
slidelock=slide==foures          # If foures=slide, lock them together
n=0                              # Rounding foures to the nearest (greater) power of 2
while (2**n)<foures:
   n+=1
foures=2**n
if slidelock:
   slide=foures
else:
   plot_on=False
   print 'No .plotd file can be produced!'
   if not spec_on:
      print 'Aborting!'
      pan.signoff()
      exit()
print 'Fourier window length rounded to 2^'+str(n)+'s ('+str(foures)+'s)!'
print 'Fourier window separation rounded to',str(slide)+'s!'
print ''

#-----Rescaling GTI--------------------------------------------------------------------

for j in pan.eqrange(gti):
   gti[j]=gti[j][0]-sttim,gti[j][1]-sttim

#-----Setting up power spectra---------------------------------------------------------

ndat=int(max(times)/bsz)
datres=int(foures/bsz)                      # Work out how many data points corresponds
                                            #   to the user given time interval
                                            #   'foures'
stpres=int(slide/bsz)                       # Work out how many data points corresponds
                                            #   to the user given time separation
                                            #   'slide'
numstep=(ndat/stpres)                       # Calculate how many intervals of 'datres'
                                            #   can be divided into the data length
while ((numstep-1)*slide)+foures>max(times):
   numstep-=1                               # Way to make sure the final bin doesn't
                                            #   exceed the data time limit
print 'Analysing data...'
print ''
fourgrlin=[]                                # Set up matrix
bad=0                                       # Counter to count ranges which fall out of
                                            #   the GTIs
good=[]                                     # Array to keep track of which ranges were
                                            #   good
prates=[]                                   # 'Peak Rates'
trates=[]                                   # 'Trough Rates'
rates=[]                                    # Array of count rates to be populated
npcus=[]                                                                 
t=arange(0,foures+bsz,bsz)                  # Setting up SpecAngel resolution time
                                            #   series per Fourier bin
tp=arange(0,foures+bsz*spcbinfac,bsz*spcbinfac) # Setting up SpecAngel coarse
                                                #   resolution time series
tc=arange(0,foures+bsz*ptdbinfac,bsz*ptdbinfac) # Setting up PlotDemon coarse
                                                #   resolution time series per
                                                #   Fourier bin
ta=arange(0,(foures*numstep),bsz*ptdbinfac)     # Setting up PlotDemon resolution
                                                #   full time series

#-----Populating power spectra---------------------------------------------------------

fullhist=[]                                     # Create empty flux array to pass to
                                                #   plotdemon
fullerrs=[]
tcounts=0                                       # Initiate photon counter
pcus=None
if not bin_dat:
   for step in range(numstep):                              ## For every [foures]s
                                                            ##   interval in the data:
      stpoint=step*slide                                       # Calculate the start
                                                               #   point of the
                                                               #   interval
      edpoint=stpoint+foures                                   # Calculate the endpoint
                                                               #   of the interval
      in_gti=False                                             # Assume the subrange is
                                                               #   not in the GTI
      for j in pan.eqrange(gti):
         if gti[j][0]<=stpoint<edpoint<=gti[j][1]: in_gti=True # Change in_gti flag if
                                                               #   this range is wholly
                                                               #   within one GTI
      mask=times>=stpoint
      datrow=times[mask]                                       # Take all photons in
                                                               #   the event data which
                                                               #   occurred after the
                                                               #   start point
      wrdrow=inst.getwrdrow(pcwrds,mask,event[1].header['DATAMODE'])
      mask=datrow<edpoint
      datrow=datrow[mask]                                      # Remove all photons
                                                               #   which occurred after
                                                               #   the end point
      wrdrow=inst.getwrdrow(wrdrow,mask,event[1].header['DATAMODE'])
      fc,null=histogram(datrow,tc+stpoint)                     # Coarsely bin this sub
                                                               #   range of event data
      fp,null=histogram(datrow,tp+stpoint)                     #   Very Coarsely bin
                                                               #   this subrange of
                                                               #   event data
      del null
      fullhist=fullhist+list(fc)
      fullerrs=fullerrs+list((array(fc)**0.5))
      if in_gti:
         f,txis=histogram(datrow,t+stpoint)                     # Bin well this sub
                                                                #   range of event data
         pcus=inst.getpcu(wrdrow,event[1].header,t_pcus=pcus)   # Count active PCUs by
                                                                #   assuming any that
                                                                #   recorded 0 events
                                                                #   in the time period
                                                                #   were inactive
         npcus.append(pcus)
         counts=sum(f)
         peak=max(fp)
         trough=min(fp)
         rates.append(float(counts)/foures)
         prates.append(float(peak)*datres/(foures*spcbinfac))
         trates.append(float(trough)*datres/(foures*spcbinfac))
         tcounts+=counts
         tsfdata=fft(f)                                         # Fourier transform the
                                                                #   interval
         tsfdata=pan.leahyn(tsfdata,counts,datres)              # Normalise to Leahy
                                                                #   Power
         good.append(True)                                      # Flag this column as
                                                                #   good
      else:
         tsfdata=zeros(datres/2)
         npcus.append(0)
         rates.append(0)
         prates.append(0)
         trates.append(0)
         good.append(False)                                     #  Flag this column as
                                                                #    bad
      fourgrlin.append(tsfdata)                                 #  Append the FT'd data
                                                                #    to the matrix
      prog=step+1
      if (prog \% 5)==0 or prog==numstep:
         print str(prog)+'/'+str(numstep)+' series analysed...' # Display progress 
                                                                #   every 5 series
   pcg=str(int(100*tcounts/float(phcts)))+'\%'
   print ''
   print str(tcounts)+'/'+str(phcts)+' ('+pcg+') photons in GTI '\
      +str((gti[0][0],gti[-1][1]))+'!'
   if tcounts==0:
      print 'Aborting!'
      pan.signoff()
      exit()
else:                                                           # Not doing Spectra for
                                                                #   Binned data just
                                                                #   yet...
   print 'Number of PCUs unknown!'
   npcus=[int(raw_input('Number of Active PCUS: '))]            # Ask the user how many
                                                                #   there are
   ta,fullhist,fullerrs=pan.binify(times,datas/ores*bsz*ptdbinfac,
      (datas**0.5)/ores*bsz*ptdbinfac,bsz)

#-----Save .speca and .plotd files-----------------------------------------------------

print ''
print 'Saving...'
print ''
filext=(filename.split('.')[-1])           # Identify file extension from the original
                                           #   filename
if filext!=filename:
   print filename
   tfilename=filename[:-len(filext)-1]     # Remove file extension, if present
   if tfilename[-1]!='.':                  # Saving extensionless files with .. in the
                                           #   path name *breaks without this*
      filename=tfilename
filename=filename+'_'+cs+'_'+str(bsz)+'s'
if plot_on:
   pfilename=pan.plotdsv(filename,ta,array(fullhist)/bsz,array(fullerrs)/bsz,tstart,
      bsz*ptdbinfac,gti,max(npcus),bgest,'False',None,flavour,cs,mission,obsdata,
      version)
   print "PlotDemon file saved to "+pfilename
else:
   print "PlotDemon file not saved."
if spec_on:
   sfilename=pan.specasv(filename,fourgrlin,good,rates,prates,trates,tcounts,
      max(npcus),bsz,bgest,foures,flavour,cs,mission,obsdata,wtype,slide,spcbinfac,
      version)
   print "SpecAngel file saved to "+sfilename
else:
   print "SpecAngel file not saved."

#-----Footer---------------------------------------------------------------------------

pan.signoff()
\end{minted}

\section{Plot Demon}

\begin{minted}[fontsize=\scriptsize]{python}
#! /usr/bin/env python

# |----------------------------------------------------------------------|
# |------------------------------PLOT DEMON------------------------------|
# |----------------------------------------------------------------------|

# Call as ./plotdemon.py FILE1 [FILE2] [FILE3] BINNING
#
# Takes 1-3 .plotd files and plots relevant astrometric plots
#
# Arguments:
#
#  FILE1
#   The absolute path to the first file to be used (generally the lowest energy band)
#
#  [FILE2]
#   The absolute path to the second file to be used
#
#  [FILE3]
#   The absolute path to the third file to be used (generally the highest energy band)
#
#  [BINNING]
#   Optional: the size, in seconds, of bins into which data will be sorted.

#-----User-set Parameters--------------------------------------------------------------

minbin=0.0078125                  # The minimum bin size the code is allowed to attempt
                                  #   to use.  This can prevent long hang-ups
version=4.3                       # The version of PlotDemon
cbin=32.0                         # The number of bins to use when calculating
                                  #   inhomonogeneity in circfold

#-----Welcoming Header-----------------------------------------------------------------

print ''
print '-------Running Plot Demon: J.M.Court, 2014------'
print ''

#-----Importing Modules----------------------------------------------------------------

try:
   import sys,os,imp
   import pylab as pl
   import pan_lib as pan
   import numpy as np
   import scipy.signal as sig
   import scipy.optimize as optm
   from math import pi
except ImportError:
   print 'Modules missing!  Aborting!'
   print ''
   print '------------------------------------------------'
   print ''
   exit()
try:
   imp.find_module('PyAstronomy')                         # Check if PyAstronomy exists
   module_pyastro=True
except ImportError:
   module_pyastro=False

#-----Opening Files--------------------------------------------------------------------

args=sys.argv                                          # Fetching arguments; softest
                                                       #    energy band first please
pan.argcheck(args,1)
try:
   float(args[-1])                                     # If the final argument can be
                                                       #   converted to integer, assume
                                                       #   user intends it as a binning 
   isbininp=True                                       # "IS BINsize given as an
                                                       #   INPut?"
except:
   isbininp=False
nfiles=max(len(args)-isbininp-1,1)                     # Fetch number of infiles (total
                                                       #   args minus one or two iff
                                                       #   binsize given)
if nfiles>3: nfiles=3
if len(args)<2:
   file1=raw_input('Filename: ')
else:
   file1=args[1]
isplotd1=pan.filenamecheck(file1,'plotd',continu=True) # Work out whether input file is
                                                       #   a plotdemon file or a csv
ch={}                                                  # Save channel info in a library
bg1=0
bg2=0
bg3=0
print 'Opening',file1                                  # Opening file 1
x1r,y1r,ye1r,tst1,bsz1,gti,pcus1,bg1,bsub1,bdata1,flv1,
   ch[1],mis1,obsd1,v1=pan.pdload(file1,isplotd1)
y1r=y1r/float(pcus1)                                   # Normalising flux by dividing
                                                       # by the number of active PCUs
                                                       # and the binsize
ye1r=ye1r/float(pcus1)
flavour=flv1
if flavour=='':
   qflav=''
else:
   qflav=' "'+flavour+'"'
if nfiles>1:
   file2=args[2]
   isplotd2=pan.filenamecheck(file2,'plotd',continu=True)
   print 'Opening',file2                               # Opening file 2
   x2r,y2r,ye2r,tst2,bsz2,gti2,pcus2,bg2,bsub2,bdata2,flv2,ch[2],mis2,obsd2,v2\
     =pan.pdload(file2,isplotd2)
   y2r=y2r/float(pcus2)                               # Normalising flux by dividing
                                                      # by the number of active PCUs
                                                      # and the binsize
   ye2r=ye2r/float(pcus2)
else: x2r=y2r=ye2r=tst2=bsz2=None
if nfiles>2:
   file3=args[3]
   isplotd3=pan.filenamecheck(file3,'plotd',continu=True)
   print 'Opening',file3                              # Opening file 3
   x3r,y3r,ye3r,tst3,bsz3,gti3,pcus3,bg3,bsub3,bdata3,flv3,ch[3],mis3,obsd3,v3\
     =pan.pdload(file3,isplotd3)
   y3r=y3r/float(pcus3)                               # Normalising flux by dividing
                                                      # by the number of active PCUs
                                                      # and the binsize
   ye3r=ye3r/float(pcus3)                             
else: x3r=y3r=ye3r=tst3=bsz3=None
bg=(bg1+bg2+bg3)/float(pcus1)
xit1=x1r[-1]
if nfiles>1:
   xit2=x2r[-1]
else:
   xit2=None
if nfiles==3:
   xit3=x3r[-1]
else:
   xit3=None
oet=max(xit1,xit2,xit3)                               # Fetch the observation end time
mint=0                                                # Save original start and end-
                                                      # points for use in clipping
maxt=oet
if nfiles>1:                                          # Checking that start-times of
                                                      # files 1 & 2 match
   if tst1!=tst2:
      if tst1>tst2:
         while x1r[0]+tst1>x2r[0]+tst2:               # Hack data off of the start of
                                                      # file 2 until its startpoint
                                                      # matches file 1
            if len(x2r)==0:
               print 'Times domains for files 1 & 2 do not overlap!  Aborting!'
               pan.signoff()
               exit()
            x2r=np.delete(x2r,0)
            y2r=np.delete(y2r,0)
            ye2r=np.delete(ye2r,0)
         if tst1+x1r[0]!=tst2+x2r[0]:
            print 'Starting times for files 1 & 2 do not match!  Aborting!'
                                                      # If this overshoots, give up
            pan.signoff()
            exit()
         else:
            tst2+=x2r[0]                              # Amend new start time
            x2r=x2r-x2r[0]
      else:
         while x2r[0]+tst2>x1r[0]+tst1:               # Or Hack data off of the start
                                                      # of file 1 until its startpoint
                                                      # matches file 2
            if len(x1r)==0:
               print 'Times domains for files 1 & 2 do not overlap!  Aborting!'
               pan.signoff()
               exit()
            x1r=np.delete(x1r,0)
            y1r=np.delete(y1r,0)
            ye1r=np.delete(ye1r,0)
         if tst1+x1r[0]!=tst2+x2r[0]:
            print 'Starting times for files 1 & 2 do not match!  Aborting!'
            pan.signoff()
            exit()
         else:
            tst1+=x1r[0]
            x1r=x1r-x1r[0]
if nfiles>2:                                          # Checking that start-times of
                                                      # files 1 & 3 match (and thus 2 &
                                                      #  3 also match)
   if tst1!=tst3:
      if tst1>tst3:
         while x1r[0]+tst1>x3r[0]+tst3:               # Hack data off of the start of
                                                      # file 3 until its startpoint
                                                      # matches file 1
            if len(x3r)==0:
               print 'Times domains for files 1 & 3 do not overlap!  Aborting!'
               pan.signoff()
               exit()
            x3r=np.delete(x3r,0)
            y3r=np.delete(y3r,0)
            ye3r=np.delete(ye3r,0)
         if tst1+x1r[0]!=tst3+x3r[0]:
            print 'Starting times for files 1 & 3 do not match!  Aborting!'
            pan.signoff()
            exit()
         else:
            tst3+=x3r[0]                              # Amend new start time
            x3r=x3r-x3r[0]
      else:
         while x3r[0]+tst3>x1r[0]+tst1:               # Or Hack data off of the start
                                                      # of files 1 & 2 until their
                                                      # startpoint matches file 3
            if len(x1r)==0:
               print 'Times domains for files 1 & 3 do not overlap!  Aborting!'
               pan.signoff()
               exit()
            x1r=np.delete(x1r,0)
            y1r=np.delete(y1r,0)
            ye1r=np.delete(ye1r,0)
            x2r=np.delete(x2r,0)
            y2r=np.delete(y2r,0)
            ye2r=np.delete(ye2r,0)
         if tst1+x1r[0]!=tst3+x3r[0]:
            print 'Starting times for files 1 & 3 do not match!  Aborting!'
            pan.signoff()
            exit()
         else:
            tst1+=x1r[0]
            x1r=x1r-x1r[0]

#-----Binning--------------------------------------------------------------------------

if isbininp:
   binning=float(args[-1])                            # Collect binsize input if given
else:
   while True:                                        # Keep asking until good response
                                                      # is given
      try:
         binning=float(raw_input("Enter bin size (s): "))
                                                      # Ask for binsize in dialogue box
         break
      except:
         print 'Invalid bin size input!'
if minbin>max(binning,bsz1,bsz2,bsz3):
   print 'Warning!  User-entered bin is smaller than the minimum!'
   print 'Minimum can be changed in the user-input section of this code'
binning=max(binning,bsz1,bsz2,bsz3,minbin)            # Prevent overbinning by setting
                                                      # minimum binning to the maximum
                                                      # of the binnings of the files
print ''
print 'Bin size='+str(binning)+'s'  
print 'Binning File 1...'
x1,y1,ye1=pan.binify(x1r,y1r,ye1r,binning)            # Bin File 1 using 'binify' in
                                                      # pan_lib
if nfiles>1:
   print 'Binning File 2...'
   x2,y2,ye2=pan.binify(x2r,y2r,ye2r,binning)         # Bin File 2 using 'binify' in
                                                      # pan_lib
   if nfiles>2:
      print 'Binning File 3...'
      x3,y3,ye3=pan.binify(x3r,y3r,ye3r,binning)      # Bin File 3 using 'binify' in
                                                      # pan_lib
print 'Binning complete!'
print ''
wrongsize=False
x3l=len(x1)                                           # Fix to make this work for 2 or
                                                      # 3 mismatched files

#-----Force file lengths to match------------------------------------------------------

if nfiles>1:                                          # Checking file lengths match
   if len(x1)!=len(x2):
      print 'Warning!  Files 1&2 of different lengths!'
      wrongsize=True
if nfiles==3:
   if len(x1)!=len(x3):
      print 'Warning!  Files 1&3 of different lengths!'
      wrongsize=True
      x3l=len(x3)
if wrongsize:                                         # Forcing file lengths to match
                                                      # if possible
   print 'Attempting to crop files...'
   mindex=min(len(x1),len(x2),x3l)-1
   if x1[mindex]!=x2[mindex]:
      print 'Cannot crop, aborting!'
      pan.signoff()
      exit()
   if nfiles==3:
      if x1[mindex]!=x3[mindex]:
         print 'Cannot crop, aborting!'
         pan.signoff()
         exit()
   mindex+=1
   x1=x1[:mindex]
   y1=y1[:mindex]
   ye1=ye1[:mindex]
   x2=x2[:mindex]
   y2=y2[:mindex]
   ye2=ye2[:mindex]
   if nfiles==3:
      x3=x3[:mindex]
      y3=y3[:mindex]
      ye3=ye3[:mindex]
   print 'Cropped succesfully!'
   print ''
   
#-----Fetch GTI Mask-------------------------------------------------------------------

print 'Fetching GTI mask...'
def getmask(xarr,gtis):
   if gtis is None:
       return np.array([True]*len(xarr))              # Assume all is in gtis if
                                                      # loaded from csv
   else:
       return pan.gtimask(xarr,gtis)
gmask=getmask(x1,gti)                                 # A mask to blank values that
                                                      # fall outside of the GTIs
if nfiles>1:
   gmask2=getmask(x2,gti2)                            # 'And' masks for different files
   gmask=gmask&gmask2
if nfiles>2:
   gmask3=getmask(x3,gti3)    
   gmask=gmask&gmask3
print str(int(100*sum(gmask)/len(gmask)))+'% of data within GTI!'
print ''

#-----Fetch Colours--------------------------------------------------------------------

def colorget(verbose=True):                           # Define colorget to easily re-
                                                      # obtain colours if base data is
                                                      # modified
   if verbose:
      print 'Analysing Data...'
   times=x1[gmask]
   timese=np.zeros(len(times))
   ys={}
   yes={}
   col={}
   cole={}
   if nfiles==1:                                      # If only one file given, flux
                                                      # and flux_error are just the
                                                      # flux and error of this one file
      flux=y1[gmask]                                  # Use gmask to clip out the areas
                                                      # outside of GTI
      fluxe=ye1[gmask]
   elif nfiles==2:
      flux,fluxe,ys,yes,col,cole=pan.pdcolex2(y1,y2,ye1,ye2,gmask)
                                                      # Get 2/1 and 1/2 colour info
                                                      # using PDColEx in pan_lib
   elif nfiles==3:
      flux,fluxe,ys,yes,col,cole=pan.pdcolex3(y1,y2,y3,ye1,ye2,ye3,gmask)
                                                      # Get ALL colour values with 3D
                                                      # PDColEx
   else:
      print 'Error!  Too much data somehow.'          # This warning should never come
                                                      # up...
      pan.signoff()
      exit()
   return times,timese,flux,fluxe,ys,yes,col,cole
times,timese,flux,fluxe,ys,yes,col,cole=colorget()    # Use colorget
print 'Done!'
print ''

#-----Setting up plot environment------------------------------------------------------

show_block=False                                      # Do not force plots to stay open
                                                      # by default
plotopt=''
es=True                                               # Options to keep track of what
                                                      # form the data is in.  'es':
                                                      # with error bars.
cs=False                                              # 'cs' with colour key
ls=False                                              # 'ls' with delineation
folded=False                                          # 'folded' has been folded over
                                                      # some period
saveplots=False
def doplot(x,xe,y,ye,ovr=False,ft='-k',per2=False):   # Defining short function to
                                                      # determine whether errorbars are
                                                      # needed on the fly
                                                      # 'ovr' allows to override colour
                                                      # and line options so lightcurves
                                                      # can be made differently
   if ovr: formst=ft                                  # If override given, accept input
                                                      # format; if none given, just
                                                      # plot lines
   elif ls: formst='-ok'                              # If deLineate mode on, connect
                                                      # points with lines and mark
                                                      # points
   else: formst='ok'                                  # If neither deLineate nor
                                                      # override on, just plot points.
   if ls and not ovr:
      plotx=np.append(x,x[0])
      ploty=np.append(y,y[0])
      plotxe=np.append(xe,xe[0])
      plotye=np.append(ye,ye[0])
   elif per2 and ovr:
      plotx=np.append(x,x+1.0)
      ploty=np.append(y,y)
      plotxe=np.append(xe,xe)
      plotye=np.append(ye,ye)
      pl.axvline(1,color='0.7',linestyle=':')      
   else:
      plotx=x
      ploty=y
      plotxe=xe
      plotye=ye
   if cs and not ovr:                                 # If coloured mode on, colour
                                                      # first 5 data points unless
                                                      # override given
      if len(x)<5:                                    # Abort if less than 5 data
                                                      # points present
         print 'Not enough data to colour!'
      else:
         pl.plot(x[0],y[0],'or',zorder=1)             # Plot a round marker over each
                                                      # of the first five points with
                                                      # colour ascending red->blue
         pl.plot(x[1],y[1],'oy',zorder=2)
         pl.plot(x[2],y[2],'og',zorder=3)
         pl.plot(x[3],y[3],'oc',zorder=4)
         pl.plot(x[4],y[4],'ob',zorder=5)
   if es:
      pl.errorbar(plotx,ploty,xerr=plotxe,yerr=plotye,fmt=formst,zorder=0)
                                                      # Plot errorbar plot if errors
                                                      # turned on
   else:
      pl.plot(plotx,ploty,formst,zorder=0)            # Else plot regular graph
def plot_save(saveplots,show_block):                  # Add a function to redirect all
                                                      # show calls to savefigs if
                                                      # toggled
   if saveplots:
      pl.savefig(raw_input('Save plot as: '))
      print 'Plot saved!'
   else:
      pl.show(block=show_block)
def burstplot(key,text,units):
   if bursts is None:
      print 'No burst data to plot!  Run "burst get" first!'
      return
   print 'Plotting Histogram of Burst '+text+'...'
   pl.figure()
   pl.hist(bursts[key],bins=np.arange(min(bursts[key]),max(bursts[key]),
   (max(bursts[key])-min(bursts[key]))/21.0))
   pl.xlabel(text,'('+units+')')
   pl.ylabel('Frequency')
   pl.title('Histogram of Burst '+text)
   plot_save(saveplots,show_block)
fldtxt=''
bursts=None
burst_alg='cubic spline'
flux_axis=r'Flux (cts s$^{-1}$ PCU$^{-1}$)'
time_n=0                                              # Normalise dump time

#-----User Menu------------------------------------------------------------------------

def give_inst():                                      # Define printing this list of
                                                      # instructions as a function
   print 'COMMANDS: Enter a command to manipulate data.'
   print ''
   print 'DATA:'
   print '* "rebin" to reset the data and load it with a different binning.'
   print '* "clip" to clip the data.'
   print '* "norm time" to renormalise the times by the start time of the data'
   print '* "mask" to remove a range of data.'
   print '* "rms" to return the fractional rms of the data.'
   print '* "fold" to fold data over a period of your choosing'+(' (requires PyAstron'+
         'omy module!)' if not module_pyastro else '')+'.'
   print '* "autofold" to automatically seek a period over which to fold data'+(' (re'+
         'quires PyAstronomy module!)' if not module_pyastro else '')+'.'
   print '* "varifold" to fold over a non-constant period using an algorithm optimise'+
         'd for high-amplitude quasi-periodic flares.'
   print '* "plot bursts" to plot the results of the peak-finding algorithm used in v'+
         'arifold.'
   print ''
   print '1+ DATASET PLOTS:'
   print '* "lc" to plot a simple graph of flux over time.'
   print '* "bg" to plot background over time, if background has been estimated for t'+
         'hese files.'
   print '* "animate" to create an animation of the lightcurve as the binning is incr'+
         'eased.'
   print '* "circanim" to create an animation of the lightcurve circularly folded as '+
         'the period is increased.'
   print '* "lombscargle" to create a Lomb-Scargle periodogram of the lightcurve.'
   print '* "autocor" to plot the auto-correlation function.'
   print '* "rmsflux" to plot the rms-flux relationship of the data.'
   if nfiles>1:                                       # Only display 2-data-set inst-
                                                      # -ructions if 2+ datasets given
      print ''
      print '2+ DATASET PLOTS:'
      print '* "hardness21" to plot a hardness/time diagram of file2/file1 colour ove'+
            'r time.'
      print '* "hardness12" to plot a hardness/time diagram of file1/file2 colour ove'+
            'r time.'
      print '* "hid21" to plot a hardness-intensity diagram of file2/file1 colour aga'+
            'inst total flux.'
      print '* "hid12" to plot a hardness-intensity diagram of file1/file2 colour aga'+
            'inst total flux.'
      print '* "calcloop21" to return the probability of a null hysteresis in the 12 '+
            'HID.'
      print '* "col21" to plot file2/file1 colour against time.'
      print '* "col12" to plot file1/file2 colour against time.'
      print '* "band" to plot the lightcurve of a single energy band.'
      print '* "bands" to plot lightcurves of all bands on adjacent axes.'
      print '* "xbands" to plot lightcurves of all bands on the same axes.'
      print '* "compbands21" to plot lightcurves of bands 2 and 1 against each other.'
      print '* "crosscor21" to plot the cross-correlation function of band 1 with ban'+
            'd 2.'
      print '* "timeres crosscor21" to plot the time-resolved cross-correlation funct'+
            'ion of band 1 with band 2' 
      print '* "all" to plot all available data products.'
   if nfiles==3:                                      # Only display 3-data-set inst-
                                                      # -ructions if 3 datasets given
      print ''
      print '3 DATASET PLOTS:'
      print '* "hardness32" to plot a hardness/time diagram of file3/file2 colour ove'+
            'r time.'
      print '* "hardness23" to plot a hardness/time diagram of file2/file3 colour ove'+
            'r time.'
      print '* "hardness31" to plot a hardness/time diagram of file3/file1 colour ove'+
            'r time.'
      print '* "hardness13" to plot a hardness/time diagram of file1/file3 colour ove'+
            'r time.'
      print '* "hid32" to plot a hardness-intensity diagram of file3/file2 colour aga'+
            'inst total flux.'
      print '* "hid23" to plot a hardness-intensity diagram of file2/file3 colour aga'+
            'inst total flux.'
      print '* "calcloop32" to return the probability of a null hysteresis in the 32 '+
            'HID.'
      print '* "hid31" to plot a hardness-intensity diagram of file3/file1 colour aga'+
            'inst total flux.'
      print '* "hid13" to plot a hardness-intensity diagram of file1/file3 colour aga'+
            'inst total flux.'
      print '* "calcloop31" to return the probability of a null hysteresis in the 31 '+
            'HID.'
      print '* "col32" to plot file3/file2 colour against time.'
      print '* "col23" to plot file2/file3 colour against time.'
      print '* "col31" to plot file3/file1 colour against time.'
      print '* "col13" to plot file1/file3 colour against time.'
      print '* "compbands31" to plot lightcurves of bands 3 and 1 against each other.'
      print '* "compbands32" to plot lightcurves of bands 3 and 2 against each other.'
      print '* "ccd" to plot a colour-colour diagram (3/1 colour against 2/1 colour).'
      print '* "timeres crosscor31" to plot the time-resolved cross-correlation funct'+
            'ion of band 3 with band 1'
      print '* "timeres crosscor32" to plot the time-resolved cross-correlation funct'+
            'ion of band 3 with band 2'
      print '* "crosscor31" to plot the cross-correlation function of band 3 with ban'+
            'd 1.'
      print '* "crosscor32" to plot the cross-correlation function of band 3 with ban'+
            'd 2.'
   print ''
   print 'BURST ANALYSIS:'
   print '* "burst get" to interactively extract burst data for analysis.'
   print '* "burst peaks" for a histogram of peak heights of extracted bursts.'
   print '* "burst risetimes" for a histogram of rise times of extracted bursts.'
   print '* "burst falltimes" for a histogram of fall times of extracted bursts.'
   print '* "burst lengths" for a histogram of durations of extracted bursts.'
   print '* "burst help" for further information on burst analysis.'
   print ''
   print 'SAVING DATA TO ASCII:'
   print '* "export" to dump the lightcurve and colour data into an ASCII file.'
   print '* "bgdump" to export background lightcurve to an ASCII file.'
   print '* "timenorm" to toggle absolute or relative time values on x-axis.'
   print ''
   print 'TOGGLE OPTIONS:'
   print '* "errors" to toggle whether to display errors in plots.'
   print '* "lines" to toggle lines joining points in graphs.'
   print '* "ckey" to toggle colour key (red-blue) for the first five points in all p'+
         'lots.'
   print '* "save" to save to disk any plots which would otherwise be shown.'
   print ''
   print 'ADVANCED OPTIONS:'
   print '* "burstalg" to select algorithm for finding pulse peaks in lightcurve.'
   print ''
   print 'OTHER COMMANDS:'
   print '* "info" to display a list of facts and figures about the current PlotDemon'+
         ' session.'
   print '* "reflav" to rewrite the flavour text used for graph titles.'
   print '* "help" or "?" to display this list of instructions again.'
   print '* "quit" to quit.'

#give_inst()                                          # Print the list of instructions
print ''
print ' --------------------'

#-----Entering Interactive Mode--------------------------------------------------------

while plotopt not in ['quit','exit']:                 # If the previous command given
                                                      # was not quit, continue
   print ''
   plotopt=raw_input('Give command [? for help]: ').lower() # Fetch command from user
   print ''

   #-----Aliasing options--------------------------------------------------------------

   if plotopt=='shid':                                # 'shid' refers to the 2/1 HID
      plotopt='hid21'
   elif plotopt=='hhid':                              # 'hhid' refers to the 3/1 HID
      plotopt='hid31'
   elif plotopt=='hid' and nfiles==2:
      plotopt='hid21'                                 # 'hid' refers to the 2/1 HID if
                                                      # that is the only HID available

   #-----Hidden 'stick' option---------------------------------------------------------

   if plotopt=='stick':                               # For use when scripting with
                                                      # Plotdemon.  If turned on, this
                                                      # causes all plots to block when
                                                      # shown.
      show_block=not show_block
      if show_block:
         print 'Sticky Plots on!'
      else:
         print 'Sticky Plots off!'

   #-----'save' option-----------------------------------------------------------------

   elif plotopt=='save':                              # Causes a plot to be saved when
                                                      # it would otherwise have been
                                                      # shown
      saveplots=not saveplots
      if saveplots:
         print 'Plot saving on!'
      else:
         print 'Plot saving off!'

   #-----'rebin' option----------------------------------------------------------------

   elif plotopt=='rebin':                             # Rebin data
      bursts=None                                     # Remove burst data
      fldtxt=''
      while True:                                     # Keep asking until a good
                                                      # response is given
         try:
            binning=float(raw_input("Enter bin size (s): "))      # Ask for binsize in
                                                                  #dialogue box
            assert binning>=minbin
            break
         except:
            print 'Invalid bin size input!'
      print 'Binning File 1...'
      x1,y1,ye1=pan.binify(x1r,y1r,ye1r,binning)          # Bin File 1 using 'binify'
                                                          # in pan_lib
      if nfiles>1:
         print 'Binning File 2...'
         x2,y2,ye2=pan.binify(x2r,y2r,ye2r,binning)       # Bin File 2 using 'binify'
                                                          # in pan_lib
         if nfiles>2:
            print 'Binning File 3...'
            x3,y3,ye3=pan.binify(x3r,y3r,ye3r,binning)    # Bin File 3 using 'binify'
                                                          # in pan_lib
      if nfiles>1:                                    # Checking file lengths match
         if len(x1)!=len(x2):
            wrongsize=True
      if nfiles==3:
         if len(x1)!=len(x3):
            wrongsize=True
            x3l=len(x3)
      if wrongsize:                                   # Forcing file lengths to match
                                                      # if possible
         mindex=min(len(x1),len(x2),x3l)-1
         if x1[mindex]!=x2[mindex]:
            print 'Cannot crop, aborting!'
            pan.signoff()
            exit()
         if nfiles==3:
            if x1[mindex]!=x3[mindex]:
               print 'Cannot crop, aborting!'
               pan.signoff()
               exit()
         mindex+=1
         x1=x1[:mindex]
         y1=y1[:mindex]
         ye1=ye1[:mindex]
         x2=x2[:mindex]
         y2=y2[:mindex]
         ye2=ye2[:mindex]
         if nfiles==3:
            x3=x3[:mindex]
            y3=y3[:mindex]
            ye3=ye3[:mindex]
      gmask=getmask(x1,gti)                           # Re-establish gmask
      print 'Binning complete!'
      print ''
      times,timese,flux,fluxe,ys,yes,col,cole=colorget() # Re-get colours
      folded=False                                       # Re-allow clipping
      print 'Done!'
      print ''

   #-----'fold' Option-----------------------------------------------------------------

   elif plotopt=='fold':                              # Fold lightcurve
      bursts=None                                     # Remove burst data
      if folded:
         print 'Data already folded!  Rebin before re-folding.'
         continue
      if not module_pyastro:                          # Only attempt to fold if pyastro
                                                      # is present
         print 'PyAstronomy Module not found!  Cannot perform fold!'# Warn user they
                                                                    # cannot fold as
                                                                    # module is missing
         continue
      while True:                                     # Keep asking user until they
                                                      # give a sensible period
         try:
            period=float(raw_input('Input period to fold over (s): '))   # Fetch period
                                                                         # from user
            break
         except:
            print "Invalid period!"                   # Keep trying until they give a
                                                      # sensible input
      while True:                                     # Keep asking user until they
                                                      # put a sensible phase resolution
         try:
            phres=float(raw_input('Input phase resolution (0-1): ')) # Fetch phase
                                                                     # resolution from
                                                                     # user
            assert phres<1.0
            assert phres>0.0
            break
         except:
            print "Invalid phase resolution!"         # Keep trying until they give a
                                                      # sensible input
      x1=x1[gmask];y1=y1[gmask];ye1=ye1[gmask]        # Zeroing all data points outside
                                                      # of GTI
      x1,y1,ye1=pan.foldify(x1,y1,ye1,period,binning,phres=phres,name='ch. '+ch[1])
                                                      # Fold using foldify function
                                                      # from pan_lib
      fldtxt='Folded '
      if nfiles>1:
         x2=x2[gmask];y2=y2[gmask];ye2=ye2[gmask]     # Zeroing all data points outside
                                                      # of GTI
         x2,y2,ye2=pan.foldify(x2,y2,ye2,period,binning,phres=phres,name='ch. '+ch[2])
                                                      # Fold data of file 2 if present
      if nfiles==3:
         x3=x3[gmask];y3=y3[gmask];ye3=ye3[gmask]     # Zeroing all data points outside
                                                      # of GTI
         x3,y3,ye3=pan.foldify(x3,y3,ye3,period,binning,phres=phres,name='ch. '+ch[3])
                                                      # Fold data of file 3 if present
      gmask=np.ones(len(x1),dtype=bool)               # Re-establish gmask
      times,timese,flux,fluxe,ys,yes,col,cole=colorget()  # Re-get colours
      folded=True
      print 'Folding Complete!'
      print ''

   #-----'norm time' Option------------------------------------------------------------

   elif plotopt=='norm time':
      if folded:
         print 'Cannot renormalise time on folded data!'
         continue
      times=times-times[0]
      print 'Renormalised times!'

   #-----'autofold' Option-------------------------------------------------------------

   elif plotopt=='autofold':                          # Autofold data lightcurve
      bursts=None                                     # Remove burst data
      if folded:
         print 'Data already folded!  Rebin before re-folding.'
         continue
      if not module_pyastro:                          # Only attempt to fold if pyastro
                                                      # is present
         print 'PyAstronomy Module not found!  Cannot perform fold!'
                                                      # Warn user they cannot fold as
                                                      # module is missing
         continue
      ls_st=max(4.0/(times[-1]-times[0]),0.005)
      ls_end=0.5/binning
      lsx=np.arange(ls_st,ls_end,(ls_end-ls_st)/2500.0) # Perform Lomb-Scargle Analysis
                                                        # on the data to seek best
                                                        # period
      lsy=pan.lomb_scargle(times,flux,fluxe,lsx)
      period=1.0/(lsx[lsy.tolist().index(max(lsy))])
      while True:                                     # Keep asking user until they put
                                                      # a sensible phase resolution
         try:
            phres=float(raw_input('Input phase resolution (0-1): '))
                                                      # Get phase resolution from user
            assert phres<1.0
            assert phres>0.0
            break
         except:
            print "Invalid phase resolution!"         # Keep trying until they give a
                                                      # sensible input
      print ''
      print 'Using period of '+str(period)+'!'
      x1=x1[gmask];y1=y1[gmask];ye1=ye1[gmask]        # Zeroing all data points outside
                                                      # of GTI
      x1,y1,ye1=pan.foldify(x1,y1,ye1,period,binning,phres=phres,name='ch. '+ch[1])
                                                      # Fold using foldify function
                                                      # from pan_lib
      fldtxt='Folded '
      if nfiles>1:
         x2=x2[gmask];y2=y2[gmask];ye2=ye2[gmask]     # Zeroing all data points outside
                                                      # of GTI
         x2,y2,ye2=pan.foldify(x2,y2,ye2,period,binning,phres=phres,name='ch. '+ch[2])
                                                      # Fold data of file 2 if present
      if nfiles==3:
         x3=x3[gmask];y3=y3[gmask];ye3=ye3[gmask]     # Zeroing all data points outside
                                                      # of GTI
         x3,y3,ye3=pan.foldify(x3,y3,ye3,period,binning,phres=phres,name='ch. '+ch[3])
                                                      # Fold data of file 3 if present
      gmask=np.ones(len(x1),dtype=bool)               # Re-establish gmask
      times,timese,flux,fluxe,ys,yes,col,cole=colorget()       # Re-get colours
      folded=True
      print 'Folding Complete!'
      print ''

   #-----'varifold' Option-------------------------------------------------------------

   elif plotopt=='varifold':
      if folded:
         print 'Cannot perform burst analysis on folded data!'
         continue
      if burst_alg=='cubic spline':
         while True:
            try:
               iq_lo=float(raw_input('Low Threshold:  '))
               iq_hi=float(raw_input('High Threshold: '))
               assert iq_hi>iq_lo
               assert iq_hi<=100
               assert iq_lo>=0
               break
            except AssertionError:
               print 'Invalid Entry!  Valid entry is of the form High>Low.'
      else:
         iq_lo=0
         iq_hi=100
      while True:
         try:
            phase_res=float(raw_input('Input phase resolution (0-1): '))
            assert phase_res<1.0
            assert phase_res>0.0
            break
         except AssertionError:
            print 'Invalid Phase Resolution!'
      phases,numpeaks,flpeaks=pan.fold_bursts(times,flux,iq_hi,iq_lo,do_smooth=False,
                                              alg=burst_alg,savgol=5)
      peaksep=(times[-1]-times[0])/numpeaks
      print numpeaks,'flares identified: average separation of',str(peaksep)+'s'
      st_time=flpeaks[0]
      endtime=flpeaks[1]
      intran=np.array(range(len(times)))
      ymask1=intran>=st_time
      ymask2=intran<endtime
      ymask=ymask1&ymask2
      nbins=int(1.0/phase_res)
      print len(phases),len(x1),len(times)
      print len(gmask)
      phases=(nbins*phases[ymask]).astype(int)
      x1=x1[gmask][ymask];y1=y1[gmask][ymask];ye1=ye1[gmask][ymask]
                                                      # Removing all data points
                                                      # outside of GTI
      newx1=[]
      newy1=[]
      newye1=[]
      for i in range(nbins):
         newx1.append(float(i)/float(nbins))
         newy1.append(np.mean(y1[phases==i]))
         newye1.append((np.sum(ye1[phases==i]**2))**0.5/len(ye1[phases==i]))
      x1=np.array(newx1)
      #y1=sig.savgol(np.array(newy1),5,3)
      #ye1=sig.savgol(np.array(newye1),5,3)
      y1=np.array(newy1)
      ye1=np.array(newye1)
      if nfiles>1:
         x2=x2[gmask][ymask];y2=y2[gmask][ymask];ye2=ye2[gmask][ymask]
                                                      # Removing all data points
                                                      # outside of GTI
         newx2=[]
         newy2=[]
         newye2=[]
         for i in range(nbins):
            newx2.append(float(i)/float(nbins))
            newy2.append(np.mean(y2[phases==i]))
            newye2.append((np.sum(ye2[phases==i]**2))**0.5/len(ye2[phases==i]))
         x2=np.array(newx2)
         y2=np.array(newy2)
         ye2=np.array(newye2)
      if nfiles==3:
         x3=x3[gmask][ymask];y3=y3[gmask][ymask];ye3=ye3[gmask][ymask]
                                                      # Removing all data points
                                                      # outside of GTI
         newx3=[]
         newy3=[]
         newye3=[]
         for i in range(nbins):
            newx3.append(float(i)/float(nbins))
            newy3.append(np.mean(y3[phases==i]))
            newye3.append((np.sum(ye3[phases==i]**2))**0.5/len(ye3[phases==i]))
         x3=np.array(newx3)
         y3=np.array(newy3)
         ye3=np.array(newye3)
      gmask=np.ones(len(x1),dtype=bool)               # Re-establish gmask
      times,timese,flux,fluxe,ys,yes,col,cole=colorget()       # Re-get colours
      folded=True
      print 'Folding Complete!'
      print ''
      period='N/A'

   #-----Get GTIs----------------------------------------------------------------------

   elif plotopt=='get gtis':
      if folded:
         print 'Cannot perform burst analysis on folded data!'
         continue
      if burst_alg=='cubic spline':
         while True:
            try:
               iq_lo=float(raw_input('Low Threshold:  '))
               iq_hi=float(raw_input('High Threshold: '))
               assert iq_hi>iq_lo
               assert iq_hi<=100
               assert iq_lo>=0
               break
            except AssertionError:
               print 'Invalid Entry!  Valid entry is of the form High>Low.'
      else:
         iq_lo=0
         iq_hi=100
      while True:
         try:
            nphbins=int(raw_input('Number of phase bins: '))
            assert nphbins>1
            break
         except AssertionError:
            print 'Invalid Phase Resolution!'
      spline=pan.get_phases_intp(flux,windows=1,q_lo=iq_lo,q_hi=iq_hi,peaks=None,
                                 givespline=True)
      start_valid=times[spline.firstpeak]+tst1
      end_valid=times[spline.lastpeak]+tst1
      numpeaks=int(spline(spline.lastpeak))
      print 'Spline created, extracting phases...'
      print ''
      flnm_prefix=raw_input('Filename Prefix: ')
      gtif={}
      for i in range(nphbins):
          gtif[i]=open(flnm_prefix+'_'+str(i)+'.csv','w')
      guess=spline(0)          
      prevcut=optm.fsolve(spline,guess)[0]*binning+tst1
      for i in range(numpeaks):
          for j in range(nphbins):
              subval=i+((j+1)/float(nphbins))
              def newspline(x):
                  return spline(x)-subval
              newcut=optm.fsolve(newspline,guess)[0]*binning+tst1
              if newcut<end_valid and prevcut>start_valid:
                 gtif[j].write(str(prevcut)+','+str(newcut)+'\n')
              prevcut=newcut     
      for i in range(nphbins):
          gtif[i].close()
      print ''
      print 'GTI files written!'

   #-----'Plot Bursts' Option----------------------------------------------------------

   elif plotopt=='plot bursts':
      if folded:
         print 'Cannot perform burst analysis on folded data!'
         continue
      while True:
         try:
            q_lo=float(raw_input('Low Threshold : '))
            assert q_lo<100
            assert q_lo>0
            break
         except:
            pass
      while True:
         try:
            q_hi=float(raw_input('High Threshold: '))
            assert q_hi<100
            assert q_hi>0
            assert q_hi>=q_lo
            break
         except:
            pass
      peaks=pan.get_bursts_windowed(flux,1,q_lo=q_lo,q_hi=q_hi,smooth=False)
      pl.figure()
      doplot(times,timese,flux,fluxe,ovr=True,per2=False)
      for i in peaks:
         pl.plot([times[i]],[flux[i]],'g*',zorder=5)
      pl.axhline(np.percentile(flux,q_lo),color='b',zorder=2)
      pl.axhline(np.percentile(flux,q_hi),color='r',zorder=2)
      plot_save(saveplots,show_block)

   #-----'clip' Option-----------------------------------------------------------------

   elif plotopt=='clip':                              # Clipping data
      bursts=None                                     # Remove burst data
      if folded:
         print 'Cannot clip folded data!'
      else:
         print 'Clipping data'
         print ''
         print 'Time range is '+str(x1[0])+'s - '+str(x1[-1])+'s'
         print 'Please choose new range of data:'
         mint,maxt,srbool=pan.srinr(x1,binning,'time')# Fetch new time domain endpoints
                                                      # using srinr function from
                                                      # pan_lib
         if srbool:
            print 'Clipping...'
            x1=x1[mint:maxt]                          # Clip file 1
            y1=y1[mint:maxt]
            ye1=ye1[mint:maxt]
            if nfiles>1:
               x2=x2[mint:maxt]                       # Clip file 2
               y2=y2[mint:maxt]
               ye2=ye2[mint:maxt]
            if nfiles==3:
               x3=x3[mint:maxt]                       # Clip file 3
               y3=y3[mint:maxt]
               ye3=ye3[mint:maxt]
            gmask=getmask(x1,gti)                     # Re-establish gmask
            times,timese,flux,fluxe,ys,yes,col,cole=colorget() # Re-get colours
            print 'Data clipped!'

   #-----'mask' Option-----------------------------------------------------------------

   elif plotopt=='mask':
      bursts=None                                     # Remove burst data
      if folded:
         print 'Cannot mask folded data!'
      else:
         print 'Masking data'
         print ''
         print 'Select time range to mask: '
         mint,maxt,srbool=pan.srinr(x1,binning,'time')# Fetch time domain endpoints of
                                                      # bad window using srinr function
                                                      # from pan_lib
         if srbool:
            print 'Masking...'
            gmask[mint:maxt]=False                    # Force all values inside the bad
                                                      # window to appear as outside of
                                                      # GTIs
            times,timese,flux,fluxe,ys,yes,col,cole=colorget()    # Re-get colours
            print 'Data masked!'

   #-----'rms' Option------------------------------------------------------------------

   elif plotopt=='rms':
      if nfiles>1:                                    # If more than one file loaded,
                                                      # prompt user to select one
         if nfiles==3:
            is_band_3=', 3'
         else:
            is_band_3=''
         selected_band=raw_input('Select Energy Band [1, 2'+is_band_3+', All]: '
                                ).lower()
         if selected_band not in ['1','2','3','all']:
            print 'Invalid band!'
            continue
      else:
         selected_band='all'
      if selected_band=='1':                          # Fetch the rms
         rms=pan.rms(y1)
      elif selected_band=='2':
         rms=pan.rms(y2)
      elif selected_band=='3':
         rms=pan.rms(y3)
      else:
         rms=pan.rms(flux)
      if rms=='div0':                                 # If the mean of the data is 0,
                                                      # fail safely
         print 'Error!  Div 0 Encountered!  Aborting!'
      else:
         print 'rms =',str(rms*100)+'%'               # Otherwise, print RMS

   #-----'rmsflux' Option--------------------------------------------------------------

   elif plotopt=='rmsflux':
      try:
         trmsbin=float(raw_input('Time binning: '))
         frmsbin=float(raw_input('Rate binning: '))
      except:
         print 'Invalid value(s)!'
         continue
      ilen=int(trmsbin/binning)
      numbins=int(len(times)/ilen)
      frflux=[]
      frrms_=[]
      frrmse=[]
      for i in range(numbins):
         kst=i*ilen
         ked=(i+1)*ilen
         kflux=flux[kst:ked]
         kfluxe=fluxe[kst:ked]
         arms,armse=pan.rms(kflux,data_err=kfluxe,with_err=True)
         frrms_.append(arms)
         frrmse.append(armse)
         frflux.append(np.mean(kflux))
      frflux=np.array(frflux)
      frrms_=np.array(frrms_)*frflux
      frrmse=np.array(frrmse)*frflux
      cflux=[]
      crms_=[]
      crmse=[]
      top=int(max(frflux)/frmsbin) +1
      for i in range(0,top):
         flow=i*frmsbin
         f_hi=(i+1)*frmsbin
         mask=np.logical_and(frflux>=flow,frflux<f_hi)
         if np.sum(mask)==0:
            continue
         mask=np.logical_and(mask,np.logical_not(np.isnan(frrms_)))
         cflux.append((flow+f_hi)/2.0)
         crms_.append(np.mean(frrms_[mask]))
         crmse.append(np.sqrt(np.sum(frrmse[mask])**2 )/sum(mask))
      pl.figure()      
      pl.errorbar(cflux,crms_,yerr=crmse,fmt='x',color='0.7',zorder=20)
      pl.plot(cflux,crms_,'kx',zorder=25)
      pl.xlabel('Rate (cts s$^{-1}$')
      pl.ylabel('RMS (cts s$^{-1}$')
      plot_save(saveplots,show_block) 

   #-----'lc' Option-------------------------------------------------------------------

   elif plotopt=='lc':                                # Plot lightcurve
      taxis='Phase' if folded else 'Time (s)'
      pl.figure()
      doplot(times,timese,flux,fluxe,ovr=True,per2=folded)     # Plot flux/time using
                                                               # doplot from pan_lib
      pl.xlabel(taxis)
      pl.ylabel(flux_axis)
      pl.ylim(ymin=0)
      pl.title(fldtxt+'Lightcurve'+qflav)
      plot_save(saveplots,show_block)
      print 'Lightcurve plotted!'

   #-----'bg' Option-------------------------------------------------------------------

   elif plotopt=='bg':                                # Plot background lightcurve
      is_bdata=(bdata1!=None)
      if nfiles>=2:
         is_bdata=is_bdata and (bdata2!=None)
      if nfiles==3:
         is_bdata=is_bdata and (bdata3!=None)
      if is_bdata:                                    # Only proceed if all infiles
                                                      # have background data
         try:
            bdata_x=bdata1[0]
            bdata_y=bdata1[1]
            if nfiles>=2:
               bdata_y+=bdata2[1]
            if nfiles==3:
               bdata_y+=bdata3[1]                     # Sum counts of all <=3
                                                      # backgrounds 
            pl.figure()
            pl.plot(bdata_x,bdata_y)
            pl.xlabel('Time (s)')
            pl.ylabel(flux_axis)
            pl.ylim(ymin=0)
            pl.title(fldtxt+'Lightcurve'+qflav)
            plot_save(saveplots,show_block)
            print 'Background plotted!'
         except:
            print 'Backgrounds inconsistenly formatted!' # Abort if backgrounds are
                                                         # somehow of different lengths
      else:
         print 'Not all files have background data!'

   #-----'bgdump' Option---------------------------------------------------------------

   elif plotopt=='bgdump':                            # Export background lightcurve
      is_bdata=(bdata1!=None)
      if nfiles>=2:
         is_bdata=is_bdata and (bdata2!=None)
      if nfiles==3:
         is_bdata=is_bdata and (bdata3!=None)
      if is_bdata:                                    # Only proceed if all infiles
                                                      # have background data
         try:
            bdata_x=bdata1[0]
            bdata_y=bdata1[1]
            if nfiles>=2:
               bdata_y+=bdata2[1]
            if nfiles==3:
               bdata_y+=bdata3[1]                     # Sum counts of all <=3
                                                      # backgrounds
         except:
            print 'Backgrounds inconsistenly formatted!' # Abort if backgrounds are
                                                         # somehow of different lengths
            continue
         ofilename=raw_input('Save textfile as: ')    # Fetch filename from user
         ofil = open(ofilename, 'w')                  # Open file
         for i in range(len(bdata_x)):
            row=['0.0']*3                             # Create a row of strings reading
                                                      # 0.0, append data into it
            row[0]=str(bdata_x[i]+time_n)+' '         # Column 01: Time
            row[1]=str(bdata_y[i])+' '                # Column 02: Rate
            row[2]='\n'
            ofil.writelines(row) 
         print 'Background saved as '+ofilename+'!'
      else:
         print 'Not all files have background data!'

   #-----'export' Option---------------------------------------------------------------

   elif plotopt=='export':                            # Export lightcurve to ASCII file
      ofilename=raw_input('Save textfile as: ')       # Fetch filename from user
      ofil = open(ofilename, 'w')                     # Open file
      if folded:
         itr=[0,1]
      else:
         itr=[0]
      for ti in itr:
         for i in range(len(times)):
            row=['0.0 ']*15                           # Create a row of strings reading
                                                      #  0.0, append data into it
            row[0]=str(times[i]+time_n+ti)+' '        # Column 00: Time
            row[1]=str(timese[i])+' '                 # Column 01: Time Error
            row[2]=str(flux[i])+' '                   # Column 02: Total Flux
            row[3]=str(fluxe[i])+' '                  # Column 03: Total Flux Error
            row[4]=str(y1[gmask][i])+' '              # Column 04: Band 1 Flux
            row[5]=str(ye1[gmask][i])+' '             # Column 05: Band 1 Flux Error
            row[14]='\n'                              # Column 14: Return (so further
                                                      #  data will be appended to a new
                                                      #  line)
            if nfiles>1:                              # If 2+ bands are given:
               row[6]=str(y2[gmask][i])+' '           # Column 06: Band 2 Flux
               row[7]=str(ye2[gmask][i])+' '          # Column 07: Band 2 Flux Error
               row[10]=str(col[21][i])+' '            # Column 10: [2/1] Colour
               row[11]=str(cole[21][i])+' '           # Column 11: [2/1] Colour Error
            if nfiles==3:                             # If 3 bands are given:
               row[8]=str(y3[gmask][i])+' '           # Column 08: Band 3 Flux
               row[9]=str(ye3[gmask][i])+' '          # Column 09: Band 3 Flux Error
               row[12]=str(col[31][i])+' '            # Column 12: [3/1] Colour
               row[13]=str(cole[31][i])+' '           # Column 13: [3/1] Colour Error
            ofil.writelines(row)                      # Append row of data into open
                                                      #  file
      ofil.close()                                    # Close file
      print 'Data saved!'

   #-----'timenorm' Option-------------------------------------------------------------
   
   elif plotopt=='timenorm':
      print ''
      if time_n==0:
         time_n=tst1
         print 'Using absolute time!'
      else:
         time_n=0
         print 'Using relative time!'
      print ''

   #-----'animate' Option--------------------------------------------------------------

   elif plotopt=='animate':
      animsloc=raw_input('Folder to save images: ')
      print ''
      if os.path.exists(animsloc):                    # Create the folder
         print 'Folder "'+animsloc+'" already exists...'
      else:
         print 'Creating folder "'+animsloc+'"...'
         os.makedirs(animsloc)
      here=os.getcwd()                                # Get current working directory
                                                      # (to move back to later)
      os.chdir(animsloc)                              # Change working directory to
                                                      # animation location
      animbin=0.0025                                  # Start with an arbitrarily low
                                                      # binsize
      while animbin<max(bsz1,bsz2,bsz3,minbin):       # Find lowest allowable binsize
                                                      # of the form 0.01*2^N
         animbin=animbin*2
      anstep=1                                        # Track the number of steps taken
      dst=times[0]                                    # Fetch largest and smallest
                                                      # times to use to force same
                                                      # scale on all graphs
      det=times[-1]
      while animbin<(det-dst)/4.0:                    # Set the maximum binsize at one
                                                      # quarter of the observation
                                                      # length
         print "Creating",str(animbin)+"s binned lightcurve"
         x1,y1,ye1=pan.binify(x1r,y1r,ye1r,animbin)   # Bin File 1 using 'binify' in
                                                      # pan_lib
         if nfiles>1:
            x2,y2,ye2=pan.binify(x2r,y2r,ye2r,animbin)# Bin File 2 using 'binify' in
                                                      # pan_lib
            if nfiles>2:
               x3,y3,ye3=pan.binify(x3r,y3r,ye3r,animbin)
                                                      # Bin File 3 using 'binify' in
                                                      # pan_lib
         mina,maxa,srbool=pan.srinr(x1,binning,'time',minv=dst,maxv=det)
                                                      # Clip each individual lightcurve
         if srbool:
            x1=x1[mina:maxa]                          # Clip file 1
            y1=y1[mina:maxa]
            ye1=ye1[mina:maxa]
            if nfiles>1:
               x2=x2[mina:maxa]                       # Clip file 2
               y2=y2[mina:maxa]
               ye2=ye2[mina:maxa]
            if nfiles==3:
               x3=x3[mint:maxt]                       # Clip file 3
               y3=y3[mint:maxt]
               ye3=ye3[mint:maxt]
         gmask=getmask(x1,gti)                        # Re-establish gmask
         times,timese,flux,fluxe,ys,yes,col,cole=colorget(verbose=False)
                                                      # Re-get colours
         if anstep==1:
            if es:
              merr=max(fluxe)
            else:
              merr=0
            maxany=max(flux)+merr                     # Calculate the scale of all
                                                      # plots based on the range of the
                                                      # first plot
            minany=min(flux)-merr
            if minany<0: minany=0
         taxis='Phase' if folded else 'Time (s)'
         pl.figure()
         doplot(times,timese,flux,fluxe,ovr=True)     # Plot the graph using doplot
         pl.xlabel(taxis)
         pl.ylabel(flux_axis)
         pl.title('Lightcurve ('+str(animbin)+'s binning)')
         pl.xlim(dst,det)
         pl.ylim(max(minany,0),maxany)
         pl.savefig(str("%04d" % anstep)+'.png')      # Save the figure with leading
                                                      # zeroes to preserve order when
                                                      # int convereted to string
         pl.close()
         anstep+=1                                    # Increment the step tracker
         animbin=animbin*2                            # Double the binsize
      print 'Cleaning up...'
      os.system ("convert -delay 10 -loop 0 *.png animation.gif")
                                                      # Use the bash command 'convert'
                                                      # to create the animated gif
      x1,y1,ye1=pan.binify(x1r,y1r,ye1r,binning)      # Reset binning of File 1 using
                                                      # 'binify' in pan_lib
      if nfiles>1:
         x2,y2,ye2=pan.binify(x2r,y2r,ye2r,binning)   # Reset binning of File 2 using
                                                      # 'binify' in pan_lib
         if nfiles>2:
            x3,y3,ye3=pan.binify(x3r,y3r,ye3r,binning)# Reset binning File 3 using
                                                      # 'binify' in pan_lib
      gmask=getmask(x1,gti)                           # Re-establish gmask
      times,timese,flux,fluxe,ys,yes,col,cole=colorget(verbose=False) # Re-get colours
      print ''
      print "Animation saved to",animsloc+'/animation.gif!'
      os.chdir(here)        


   #-----'hidxy' Option----------------------------------------------------------------

   elif plotopt[:3]=='hid':                           # Plot x/y HID
      ht=plotopt[3:]                                  # Collect the xy token from the
                                                      # user
      if nfiles>1:
         if not (ht in ['12','13','21','23','31','32']):
                                                      # Check that the token is 2 long
                                                      # long and contains two different
                                                      # characters of the set [1,2,3]
            print 'Invalid command!'
            print ''
            print 'Did you mean...'
            print ''
            print 'HID options:'
            print '* "hid21" for 2/1 colour'
            print '* "hid12" for 1/2 colour'
            if nfiles==3:
               print '* "hid32" for 3/2 colour'
               print '* "hid23" for 2/3 colour'
               print '* "hid31" for 3/1 colour'
               print '* "hid13" for 1/3 colour'
         elif ('3' in ht) and (nfiles<3):
            print 'Not enough infiles for advanced HID!'
                                                      # If token contains a 3 but only
                                                      # 2 infiles are used, abort!
         else:
            h1=int(ht[0])                             # Extract numerator file number
            h2=int(ht[1])                             # Extract denominator file number
            ht=int(ht)
            pl.figure()
            doplot(col[ht],cole[ht],flux,fluxe)       # Collect colours from col
                                                      # library and plot
            pl.ylabel(flux_axis)
            pl.xlabel('('+ch[h1]+'/'+ch[h2]+') colour')
            pl.title(fldtxt+'Hardness Intensity Diagram'+qflav)
            plot_save(saveplots,show_block)
            print 'File'+str(h1)+'/File'+str(h2)+' HID plotted!'
      else:
         print 'Not enough infiles for HID!'

   #-----'calcloopxy' Option-----------------------------------------------------------

   elif plotopt[:8]=='calcloop':                      # Calculate likelihood of loop in
                                                      # HID
      ht=plotopt[8:]                                  # Collect the xy token from the
                                                      # user
      if nfiles>1:
         if not (ht in ['12','13','21','23','31','32']): 
                                                      # Check that the token is 2 long
                                                      # and contains two different
                                                      # characters of the set [1,2,3]
            print 'Invalid command!'
            print ''
            print 'Did you mean...'
            print ''
            print 'CalcLoop options:'
            print '* "calcloop21" for 2/1 HID loop calculation'
            print '* "calcloop12" for 1/2 HID loop calculation'
            if nfiles==3:
               print '* "calcloop32" for 3/2  HID loop calculation'
               print '* "calcloop23" for 2/3  HID loop calculation'
               print '* "calcloop31" for 3/1  HID loop calculation'
               print '* "calcloop13" for 1/3  HID loop calculation'
         elif ('3' in ht) and (nfiles<3):
            print 'Not enough infiles for advanced HID!'
                                                      # If token contains a 3 but only
                                                      # 2 infiles are used, abort!
         else:
            h1=int(ht[0])                             # Extract numerator file number
            h2=int(ht[1])                             # Extract denominator file number
            ht=int(ht)
            lkl=pan.calcloop(flux,col[ht],fluxe,cole[ht])
                                                      # Collect likelihood of loop from
                                                      # pan_lib
            print 'Null hypothesis (no hysteresis) probability = '+str(lkl)
      else:
         print 'Not enough infiles for HID!'

   #-----'hardnessxy' Option-----------------------------------------------------------

   elif plotopt[:8]=='hardness':                      # Plot x/y hardness/time plot
      ht=plotopt[8:]                                  # Collect the xy token from the
                                                      # user
      if nfiles>1:
         if not (ht in ['12','13','21','23','31','32']):
                                                      # Check that the token is 2 long
                                                      # and contains two different
                                                      # characters of the set [1,2,3]
            print 'Invalid command!'
            print ''
            print 'Did you mean...'
            print ''
            print 'HID options:'
            print '* "hardness21" for 2/1 hardness over time'
            print '* "hardness12" for 1/2 hardness over time'
            if nfiles==3:
               print '* "hardness32" for 3/2 hardness over time'
               print '* "hardness23" for 2/3 hardness over time'
               print '* "hardness31" for 3/1 hardness over time'
               print '* "hardness13" for 1/3 hardness over time'
         elif ('3' in ht) and (nfiles<3):
            print 'Not enough infiles for advanced HID!'
                                                      # If token contains a 3 but only
                                                      # 2 infiles are used, abort!
         else:
            h1=int(ht[0])                             # Extract numerator file number
            h2=int(ht[1])                             # Extract denominator file number
            ht=int(ht)
            pl.figure()
            doplot(times,timese,col[ht],cole[ht],ovr=True,per2=folded)
                                                      # Collect colours from col
                                                      # library and plot
            pl.ylabel('('+ch[h1]+'/'+ch[h2]+') colour')
            pl.xlabel(taxis)
            pl.title(fldtxt+'Hardness/Time plot'+qflav)
            plot_save(saveplots,show_block)
            print 'File'+str(h1)+'/File'+str(h2)+' Hardness/time diagram plotted!'
      else:
         print 'Not enough infiles for HID!'

   #-----'compbandsxy' Option----------------------------------------------------------

   elif plotopt[:9]=='compbands':                     # Plot two bands against each
                                                      # other
      ht=plotopt[9:]                                  # Collect the xy token from the
                                                      # user
      if nfiles>1:
         if not (ht in ['12','13','21','23','31','32']):
                                                      # Check that the token is 2 long
                                                      # and contains two different
                                                      # characters of the set [1,2,3]
            print 'Invalid command!'
            print ''
            print 'Did you mean...'
            print ''
            print 'HID options:'
            print '* "compbands21" to plot lightcurve 2 against 1'
            print '* "compbands12" to plot lightcurve 1 against 2'
            if nfiles==3:
               print '* "compbands32" to plot lightcurve 3 against 2'
               print '* "compbands23" to plot lightcurve 2 against 3'
               print '* "compbands31" to plot lightcurve 3 against 1'
               print '* "compbands13" to plot lightcurve 1 against 3'
         elif ('3' in ht) and (nfiles<3):
            print 'Not enough infiles for advanced lightcurve comparison!'
                                                      # If token contains a 3 but only
                                                      # 2 infiles are used, abort!
         else:
            h1=int(ht[0])                             # Extract numerator file number
            h2=int(ht[1])                             # Extract denominator file number
            pl.figure()
            doplot(ys[h1],yes[h1],ys[h2],yes[h2])     # Collect colours from col
                                                      # library and plot
            pl.ylabel(r'Band '+str(h1)+r' rate (cts s$^{-1}$ PCU$^{-1}$)')
            pl.xlabel(r'Band '+str(h2)+r' rate (cts s$^{-1}$ PCU$^{-1}$)')
            pl.title(fldtxt+'Lightcurve Comparison Diagram'+qflav)
            plot_save(saveplots,show_block)
            print 'File'+str(h1)+'/File'+str(h2)+' LC Comparison plotted!'
      else:
         print 'Not enough infiles for lightcurve comparison!'

   #-----'autocor' Option--------------------------------------------------------------

   elif plotopt=='autocor':                           # Plot autocorrelation function
      ccor=sig.correlate(flux-np.mean(flux),flux-np.mean(flux),mode='same')/
                         (len(flux)*(np.std(flux)**2))
      nlen=len(times)
      cct=range(nlen)                                 # Set up the lag axis
      cct=((np.array(cct)-nlen/2.0)*binning).tolist()
      pl.figure()
      pl.plot(cct,ccor)
      pl.xlabel('Lag (s)')
      pl.ylabel('Cross-Correlation')
      plot_save(saveplots,show_block)
      print 'Autocorrelation diagram plotted!'

   #-----'crosscor' Option-------------------------------------------------------------

   elif plotopt[:8]=='crosscor':                      # Plot cross-correlation function
      ht=plotopt[8:]                                  # Collect the xy token from the
                                                      # user
      if nfiles>1:
         if not (ht in ['12','13','21','23','31','32']):
                                                      # Check that the token is 2 long
                                                      # and contains two different
                                                      # characters of the set [1,2,3]
            print 'Invalid command!'
            print ''
            print 'Did you mean...'
            print ''
            print 'HID options:'
            print '* "crosscor21" for 2 against 1 cross-correlation'
            if nfiles==3:
               print '* "crosscor32" for 3 against 2 cross-correlation'
               print '* "crosscor31" for 3 against 1 cross-correlation'
         elif ('3' in ht) and (nfiles<3):
            print 'Not enough infiles for advanced cross-correlation!'
                                                      # If token contains a 3 but only
                                                      # 2 infiles are used, abort!
         else:
            nlen=len(times)
            h1=int(ht[0])                             # Extract file 1 number
            h2=int(ht[1])                             # Extract file 2 number
            h1a=ys[h1]                                # Extract dataset 1
            h2a=ys[h2]                                # Extract dataset 2
            if nlen%2==0:                             # Cross-correlation only gives a
                                                      # point at zero lag if an even
                                                      # number of points are input
               nlen-=1
               h1a=h1a[:-1]
               h2a=h2a[:-1]
            ccor=sig.correlate(h1a-np.mean(h1a),h2a-np.mean(h2a),mode='same')/
                               (nlen*np.std(h1a)*np.std(h2a))
            cct=range(nlen)                           # Set up the lag axis
            cct=(((np.array(cct)-nlen/2.0)*binning)+(binning/2.0)).tolist()
            pl.figure()
            pl.plot(cct,ccor)
            pl.axvline(0,linestyle=':',color='0.7')
            pl.xlabel('Band '+str(h1)+' lag (s) wrt Band '+str(h2))
            pl.ylabel('Cross-Correlation')
            plot_save(saveplots,show_block)
            print 'Band '+str(h1)+' lag against band '+str(h2)+
                  ' (cross-correlation) diagram plotted!'
      else:
         print 'Not enough infiles for cross-correlation!'

   #-----'timeres crosscor' Option-----------------------------------------------------

   elif plotopt[:16]=='timeres crosscor':             # Plot time-resolved cross-
                                                      # correlation function
      ht=plotopt[16:]                                 # Collect the xy token from the
                                                      # user
      if nfiles>1:
         if not (ht in ['12','13','21','23','31','32']):
                                                      # Check that the token is 2 long
                                                      # and contains two different
                                                      # characters of the set [1,2,3]
            print 'Invalid command!'
            print ''
            print 'Did you mean...'
            print ''
            print 'HID options:'
            print '* "timeres crosscor21" for 2 against 1 time-resolved cross-correla'+
                  'tion'
            if nfiles==3:
               print '* "timeres crosscor32" for 3 against 2 time-resolved cross-corr'+
                     'elation'
               print '* "timeres crosscor31" for 3 against 1 time-resolved cross-corr'+
                     'elation'
         elif ('3' in ht) and (nfiles<3)
            print 'Not enough infiles for advanced cross-correlation!'
                                                      # If token contains a 3 but only
                                                      # 2 infiles are used, abort!
         else:
            try:
               nlen=int(float(raw_input('Length of segments (s): '))/binning)
               if nlen%2!=0: nlen-=1
               frang=float(raw_input('Max lag (s) : '))
            except:
               print 'Not a number!  Aborting!'
               continue
            maxtims=[]
            maxlocs=[]
            n=len(times)/nlen
            matrix=[]                                 # Set up the 2d matrix
            mtimes=[]                                 # Set up the time axis
            xn=range(nlen)                            # Set up the lag axis
            xn=((np.array(xn)-nlen/2.0)*binning).tolist()
            h1=int(ht[0])                             # Extract file 1 number
            h2=int(ht[1])                             # Extract file 2 number
            for i in range(n):
               y1clip=np.array(ys[h1][nlen*i:nlen*(i+1)]) # Clip the first lightcurve
                                                          # into chunks
               y2clip=np.array(ys[h2][nlen*i:nlen*(i+1)]) # Clip the second lightcurve
                                                          # into chunks
               cclip=sig.correlate(y1clip-np.mean(y1clip),y2clip-np.mean(y2clip),
                               mode='same')/(len(y2clip)*np.std(y1clip)*np.std(y2clip))
                                                          # Do the cross-correlation
               cclip=np.array(cclip)[np.array(xn)<=frang] # Cut the matrix line to the
                                                          # user requested lag range
               txn=np.array(xn)[np.array(xn)<=frang]
               cclip=cclip[txn>-frang]
               txn=txn[txn>-frang]  
               maxc=max(cclip)
               maxloc=txn[cclip.tolist().index(maxc)]
               matrix.append(cclip)                   # Grow the matrix!
               mtimes.append(nlen*i*binning)
               maxtims.append(nlen*(i+0.5)*binning)
               maxlocs.append(maxloc)
            pl.figure()
            pl.pcolor(np.array(mtimes+[mtimes[-1]+binning*nlen])-binning/2.0,txn,
                      np.array(matrix).T)
            pl.xlabel('Time (s)')
            pl.ylabel('Band '+str(h1)+' lag (s) wrt Band '+str(h2))
            pl.plot(maxtims,maxlocs,':k',label='Peak lag')
            pl.legend()
            plot_save(saveplots,show_block)
            print 'Band '+str(h1)+' lag against band '+str(h2)+' (time-resolved cross'+
                  '-correlation) diagram plotted!'
      else:
         print 'Not enough infiles for cross-correlation!'

   #-----'colxy' Option----------------------------------------------------------------

   elif plotopt[:3]=='col':                           # Plot x/y colour/t
      ht=plotopt[3:]                                  # Collect the xy token from the
                                                      # user
      if nfiles>1:
         if not (ht in ['12','13','21','23','31','32']):
                                                      # Check that the token is 2 long
                                                      # and contains two different
                                                      # characters of the set [1,2,3]
            print 'Invalid command!'
            print ''
            print 'Did you mean...'
            print ''
            print 'Col/t plot options:'
            print '*"col21" for 2/1 colour'
            print '*"col12" for 1/2 colour'
            if nfiles==3:
               print '*"col32" for 3/2 colour'
               print '*"col23" for 2/3 colour'
               print '*"col31" for 3/1 colour'
               print '*"col13" for 1/3 colour'
         elif ('3' in ht) and (nfiles<3):
            print 'Not enough infiles for advanced Col/t plot!'
                                                      # If token contains a 3 but only
                                                      # 2 infiles are used, abort!
         else:
            taxis='Phase' if folded else 'Time (s)'
            h1=int(ht[0])                             # Extract numerator file number
            h2=int(ht[1])                             # Extract denominator file number
            ht=int(ht)
            pl.figure()
            doplot(times,timese,col[ht],cole[ht],ovr=True,per2=folded)
                                                      # Collect colours from col
                                                      # library and plot
            pl.xlabel(taxis)
            pl.ylabel('('+ch[h1]+'/'+ch[h2]+') colour')
            pl.ylim(ymin=0)
            pl.title(fldtxt+'Colour over Time Diagram'+qflav)
            plot_save(saveplots,show_block)
            print 'File'+str(h1)+'/File'+str(h2)+' Colour over Time Diagram plotted!'
      else:
         print 'Not enough infiles for Col/t plot!'

   #-----'ccd' Option------------------------------------------------------------------

   elif plotopt=='ccd':                               # Plot Colour-Colour diagram
      if nfiles==3:
         pl.figure()
         doplot(col[31],cole[31],col[21],cole[21])
         pl.xlabel('('+ch[2]+'/'+ch[1]+') colour')
         pl.ylabel('('+ch[3]+'/'+ch[1]+') colour')
         pl.xlim(0,2)
         pl.ylim(0,2)
         pl.title(fldtxt+'Colour-Colour Diagram'+qflav)
         plot_save(saveplots,show_block)
         print 'CCD plotted!'
      else:
         print 'Not enough infiles for CCD!'

   #-----'all' Option------------------------------------------------------------------

   elif plotopt=='all':
      pl.figure()
      if   nfiles==3: colexp=2; rowexp=2; gexp=4      # If 3 files given, 4 graphs will
                                                      # be plotted in a 2x2 grid
      elif nfiles==2: colexp=1; rowexp=2; gexp=2      # If 2 files given, 2 grapgs will
                                                      # be plotted in a 2x1 grid
      else:           colexp=1; rowexp=1; gexp=1      # If 1 file given, only one graph
                                                      # can be plotted
      print 'Plotting Lightcurve...'
      taxis='Phase' if folded else 'Time (s)'
      pl.subplot(rowexp,colexp,1)                     # Create subplot in the first slot
      doplot(times,timese,flux,fluxe,ovr=True)        # Always plot the lightcurve
      pl.xlabel(taxis)
      pl.ylabel(flux_axis)
      pl.ylim(ymin=0)
      pl.title(fldtxt+'Lightcurve'+qflav)
      if nfiles>1:                                    # If 2+ files given, plot 2+ file
                                                      # data products
         print 'Plotting Soft Hardness-Intensity Diagram...'
         pl.subplot(rowexp,colexp,2)                  # Create subplot in the second
                                                      # slot
         doplot(col[21],cole[21],flux,fluxe)          # Plot Soft HID
         pl.xlim(0,2)
         pl.ylim(0,300)
         pl.ylabel(flux_axis)
         pl.xlabel('('+ch[2]+'/'+ch[1]+') colour')
         pl.title(fldtxt+'Soft HID'+qflav)
      if nfiles==3:                                   # If 3 files given, plot 3 file
                                                      # data products
         print 'Plotting Hard Hardness-Intensity Diagram...'
         print 'Plotting Colour-Colour Diagram...'
         pl.subplot(rowexp,colexp,3)                  # Create subplot in the 3rd slot
         doplot(col[31],cole[31],flux,fluxe)          # Plot Hard HID
         pl.xlim(0,2)
         pl.ylim(0,300)
         pl.ylabel(flux_axis)
         pl.xlabel('('+ch[3]+'/'+ch[1]+') colour')
         pl.title(fldtxt+'Hard HID'+qflav)
         pl.subplot(rowexp,colexp,4)                  # Create subplot in the 4th slot
         doplot(col[31],cole[31],col[21],cole[21])    # Plot CCD
         pl.xlim(0,2)
         pl.ylim(0,2)
         pl.ylabel('('+ch[2]+'/'+ch[1]+') colour')
         pl.xlabel('('+ch[3]+'/'+ch[1]+') colour')
         pl.title(fldtxt+'CCD'+qflav)
      print ''
      plot_save(saveplots,show_block)
      print 'All products plotted!'

   #-----'band' Option-----------------------------------------------------------------

   elif plotopt=='band':                              # Plot lightcurve of individual
                                                      # band
      if nfiles==1:
         user_b_band='1'                              # Select energy band to plot
      else:
         if nfiles==3:
            is_band_3=', 3'
         else:
            is_band_3=''
         user_b_band=raw_input('Select Energy Band [1, 2'+is_band_3+']: ')
      avail_b_band=['1','2']                          # Define valid user inputs
      if nfiles==3:
         avail_b_band.append('3')                     # Add '3' as a valid input if 3
                                                      # bands present
      if user_b_band in avail_b_band:
         taxis='Phase' if folded else 'Time (s)'
         pl.figure()
         if user_b_band=='1':
            doplot(times,timese,y1[gmask],ye1[gmask],ovr=True,per2=folded)
                                                      # Plot flux/time using doplot
                                                      # from pan_lib
            b_band_name='Band 1'
         elif user_b_band=='2':
            doplot(times,timese,y2[gmask],ye2[gmask],ovr=True,per2=folded)
            b_band_name='Band 2'
         else:
            b_band_name='Band 3'
            doplot(times,timese,y3[gmask],ye3[gmask],ovr=True,per2=folded)
         pl.xlabel(taxis)                             # Format plot
         pl.ylabel(flux_axis)
         pl.ylim(ymin=0)
         pl.title(fldtxt+b_band_name+'Lightcurve'+qflav)
         plot_save(saveplots,show_block)
         print b_band_name+' lightcurve plotted!'

   #-----'bands' Option----------------------------------------------------------------

   elif plotopt=='bands':                             # Plot lightcurves of individual
                                                      # bands apart
      taxis='Phase' if folded else 'Time (s)'
      pl.figure()
      pl.subplot(nfiles,1,1) 
      doplot(times,timese,y1[gmask],ye1[gmask],ovr=True,per2=folded)
                                                      # Plot the lowest band
      pl.xlabel(taxis)
      pl.ylabel(flux_axis)
      pl.title(fldtxt+ch[1]+' Lightcurve'+qflav)
      if nfiles>1:
         pl.subplot(nfiles,1,2)
         doplot(times,timese,y2[gmask],ye2[gmask],ovr=True,per2=folded)
                                                      # Plot the second band
         pl.xlabel(taxis)
         pl.ylabel(flux_axis)
         pl.title(fldtxt+ch[2]+' Lightcurve'+flv2)
      if nfiles>2:
         pl.subplot(nfiles,1,3)
         doplot(times,timese,y3[gmask],ye3[gmask],ovr=True,per2=folded)
                                                      # Plot the third band
         pl.xlabel(taxis)
         pl.ylabel(flux_axis)
         pl.title(fldtxt+ch[3]+' Lightcurve'+flv3)
      plot_save(saveplots,show_block)
      print 'Banded lightcurves plotted!'

   #-----'xbands' Option---------------------------------------------------------------

   # 'Same axes bands'

   elif plotopt=='xbands':                            # Plot lightcurves of individual
                                                      # bands together
      donorm=raw_input('Normalise bands? : ')         # Fetch whether user wants to
                                                      # normalise
      donorm=donorm in ('y','yes') 
      taxis='Phase' if folded else 'Time (s)'
      pl.figure()
      leg=[ch[1]]                                     # Create a legend array to
                                                      # populate with channel names
      if donorm:
         n1=max(y1[gmask])
      else:
         n1=1.0
      doplot(times,timese,y1[gmask]/n1,ye1[gmask]/n1,ovr=True,ft='-b',per2=folded)
                                                      # Plot the lowest band
      if nfiles>1:
         if donorm:
            n2=max(y2[gmask])
         else:
            n2=1.0
         doplot(times,timese,y2[gmask]/n2,ye2[gmask]/n2,ovr=True,ft='-g',per2=folded)
                                                      # Plot the second band
         leg.append(ch[2])                            # Append name of second channel
                                                      # to key
      if nfiles>2:
         if donorm:
            n3=max(y3[gmask])
         else:
            n3=1.0
         doplot(times,timese,y3[gmask]/n3,ye3[gmask]/n3,ovr=True,ft='-r',per2=folded)
                                                      # Plot the third band
         leg.append(ch[3])                            # Append name of third channel to
                                                      # key
      if folded:
         pl.axvline(1,linestyle=':',color='0.7')
      pl.legend(leg)                                  # Create key on plot
      pl.xlabel(taxis)
      pl.ylabel(flux_axis)
      pl.title(fldtxt+'Lightcurve'+qflav)
      plot_save(saveplots,show_block)
      print 'Banded lightcurves plotted!'

   #-----'burst get'-------------------------------------------------------------------

   elif plotopt in ['burst get','burstget','burst_get','bursts get','burstsget',
                    'bursts_get']:
      if folded:
         print 'Cannot perform burst analsysis on folded data!'
         continue
      while True:
         try:
            iq_lo=float(raw_input('Low Threshold:  '))
            iq_hi=float(raw_input('High Threshold: '))
            assert iq_hi>iq_lo
            assert iq_hi<=100
            assert iq_lo>=0
            break
         except:
            print 'Invalid Entry!  Valid entry is of the form High>Low.'
      bursts={}
      bursts['endpoints']=pan.get_bursts(flux,q_lo=iq_lo,q_hi=iq_hi,just_peaks=False,
                                         alg=burst_alg)
      pl.figure()
      doplot(times,timese,flux,fluxe,ovr=True)        # Plot flux/time using doplot
                                                      # from pan_lib
      t_lo=np.percentile(flux,iq_lo)                  # Fetch thresholds used in
                                                      # get_bursts
      t_hi=np.percentile(flux,iq_hi)
      col_toggle=True
      for i in bursts['endpoints']:
         if col_toggle:
            col_toggle=False
            burst_colour='#c7c7c7'
         else:
            col_toggle=True
            burst_colour='#e7e7e7'
         pl.axvspan(times[i[0]],times[i[1]], facecolor=burst_colour, edgecolor='none')
      pl.plot([times[0],times[-1]],[t_lo,t_lo],'g')
      pl.plot([times[0],times[-1]],[t_hi,t_hi],'b')
      pl.legend(['Flux','Low Pass Threshold','High Pass Threshold'])
      pl.xlabel('Time (s)')
      pl.ylabel(flux_axis)
      pl.ylim(ymin=0)
      pl.title(fldtxt+'Lightcurve with Bursts Highlighted'+qflav)
      plot_save(saveplots,show_block)
      print ''
      print 'Bursts plotted!'
      print len(bursts['endpoints']),'bursts found!'
      print ''
      print 'Analysing Bursts...'
      bursts['peaks']=[]
      bursts['rises']=[]
      bursts['falls']=[]
      bursts['duras']=[]
      for endpoints in bursts['endpoints']:
         burst=flux[endpoints[0]:endpoints[1]]
         btime=times[endpoints[0]:endpoints[1]]
         peak,trough,pk_time,rise_time,fall_time=pan.eval_burst(btime,burst)
         bursts['peaks'].append(peak)
         #bursts['trghs'].append(troughs)
         bursts['rises'].append(rise_time)
         bursts['falls'].append(fall_time)
         bursts['duras'].append(btime[-1]-btime[0])
      print ''
      print 'Analysis Complete!'
      print 'Burst Products now available!'

   #-----'burst peaks'-----------------------------------------------------------------

   elif plotopt in ['burst peaks','burstpeaks','burst_peaks','bursts peaks',
                    'burstspeaks','bursts_peaks']:
      burstplot('peaks','Peak Heights','cts/s/PCU')

   #-----'burst risetimes'-------------------------------------------------------------

   elif plotopt in ['burst risetimes','burstrisetimes','burst_risetimes',
                    'bursts risetimes','burstsrisetimes','bursts_risetimes']:
      burstplot('rises','Rise Times','s')

   #-----'burst falltimes'-------------------------------------------------------------

   elif plotopt in ['burst falltimes','burstfalltimes','burst_falltimes',
                    'bursts falltimes','burstsfalltimes','bursts_falltimes']:
      burstplot('falls','Fall Times','s')

   #-----'burst lengths'---------------------------------------------------------------

   elif plotopt in ['burst lengths','burstlengths','burst_lengths','bursts lengths',
                    'burstslengths','bursts_lengths']:
      burstplot('duras','Durations','s')

   #-----'bursts help'-----------------------------------------------------------------

   elif plotopt in ['burst help','bursthelp','burst_help','bursts help','burstshelp',
                    'bursts_help']:
      print 'Help coming soon.'

   #-----'burst alg'-------------------------------------------------------------------

   elif plotopt in ['burst alg','burstalg','burst_alg','bursts alg','burstsalg',
                    'bursts_alg']:
      print 'Available Burst-Finding Algorithms:'
      print ' * CUBIC SPLINE'
      print ' * LOAD'
      print ''
      inp_burst_alg=raw_input('Select Burst Algorithm: ').lower()
      if inp_burst_alg in ('cubic spline','load'):
         burst_alg=inp_burst_alg
         print 'Burst-Finding Algorithm set to "'+burst_alg+'"!'
      else:
         print 'Invalid Burst-Finding Algorithm!'

   #-----'lombscargle' Option----------------------------------------------------------

   elif plotopt=='lombscargle':
      if folded:                                      # If data is folded, abort
         print 'Cannot perform Lomb-Scargle on folded data!'
         continue
      if nfiles==1:
         user_scargl_bands='1'                        # Select energy bands to
                                                      # LombScargle
      else:
         if nfiles==3:
            is_band_3=', 3'
         else:
            is_band_3=''
         user_scargl_bands=raw_input('Select Energy Band [1, 2'+is_band_3+', All]: '
                                     ).lower()
      ls_st=max(4.0/(times[-1]-times[0]),0.005)
      ls_end=0.5/binning
      lsx=np.arange(ls_st,ls_end,(ls_end-ls_st)/2500.0)
      avail_scargl_bands=['1','2','all']              # Define valid user inputs
      if nfiles==3:
         avail_scargl_bands.append('3')               # Add '3' as a valid input if 3
                                                      # bands present
      if user_scargl_bands in avail_scargl_bands:
         if user_scargl_bands=='1':
            lsy=pan.lomb_scargle(times,y1[gmask],ye1[gmask],lsx)
                                                      # Perform LombScargle of band 1
                                                      # using lombscargle function
                                                      # defined in header
            s_band_name='band 1'
         elif user_scargl_bands=='2':
            lsy=pan.lomb_scargle(times,y2[gmask],ye2[gmask],lsx)
                                                      # Perform LombScargle of band 2
                                                      # using lombscargle function
                                                      # defined in header
            s_band_name='band 2'
         elif user_scargl_bands=='3':
            lsy=pan.lomb_scargle(times,y3[gmask],ye3[gmask],lsx)
                                                      # Perform LombScargle of band 3
                                                      # using lombscargle function
                                                      # defined in header
            s_band_name='band 3'
         else:
            lsy=pan.lomb_scargle(times,flux,fluxe,lsx)# Perform LombScargle of all
                                                      # bands using lombscargle
                                                      # function defined in header
            s_band_name='all bands'
            if user_scargl_bands!='all':
               print 'Invalid band!  Using all.'
         pl.figure()
         pl.plot(lsx,lsy,'k')                         # Plot lombscargle
         pl.xlabel('Frequency (Hz)')
         pl.ylabel('Power')
         pl.xlim(0,max(lsx))
         pl.ylim(1,100000)
         pl.yscale('log')
         pl.title('Lomb-Scargle Periodogram of '+s_band_name+qflav)
         plot_save(saveplots,show_block)
         print ''
         print 'Lomb-Scargle Diagram of '+s_band_name+' plotted!'
      else:
         print 'Invalid energy band!'

   #-----'errors' Option---------------------------------------------------------------

   elif plotopt in ['errors','error']:                # Toggle Errors
      if es:
         es=False
         print 'Errors suppressed!'
      else:
         es=True
         print 'Errors displayed!'

   #-----'ckey' Option-----------------------------------------------------------------

   # 'Colour Key'

   elif plotopt=='ckey':                              # Toggle Colour-key
      if cs:
         cs=False
         print 'Colour key suppressed!'
      else:
         cs=True
         print 'Colour key displayed!'

   #-----'lines' Option----------------------------------------------------------------

   elif plotopt=='lines':                             # Toggle Delineation
      if ls:
         ls=False
         print 'Plot Lines suppressed!'
      else:
         ls=True
         print 'Plot Lines displayed!'

   #-----'info' Option-----------------------------------------------------------------

   elif plotopt=='info':
      dst=times[0]
      det=times[-1]
      print 'PlotDemon.py version',version
      print ''
      print nfiles,'files loaded:'
      print ''
      filn1,loca1=pan.xtrfilloc(file1)
      print 'File 1:'
      print ' Filename       = ',filn1
      print ' Location       = ',loca1
      print ' Mission        = ',mis1
      print ' Object         = ',obsd1[0]
      print ' Obs_ID         = ',obsd1[1]
      if mis1 in ['SUZAKU']:
         print ' Energy         = ',ch[1],'eV'
      else:
         print ' Channel        = ',ch[1]
      print ' Resolution     = ',str(bsz1)+'s'
      print ' BG Subtracted  = ',bsub1
      print ' No. of PCUs    = ',pcus1
      print ' Flavour        = ',flv1
      print ' FITSGenie Ver. = ',v1
      if nfiles>1:
         filn2,loca2=pan.xtrfilloc(file2)
         print ''
         print 'File 2:'
         print ' Filename       = ',filn2
         print ' Location       = ',loca2
         print ' Mission        = ',mis2
         print ' Object         = ',obsd2[0]
         print ' Obs_ID         = ',obsd2[1]
         if mis2 in ['SUZAKU']:
            print ' Energy         = ',ch[2],'eV'
         else:
            print ' Channel        = ',ch[2]
         print ' Resolution     = ',str(bsz2)+'s'
         print ' BG Subtracted  = ',bsub2
         print ' No. of PCUs    = ',pcus2
         print ' Flavour        = ',flv2
         print ' FITSGenie Ver. = ',v2
      if nfiles==3:
         filn3,loca3=pan.xtrfilloc(file3)
         print ''
         print 'File 3:'
         print ' Filename       = ',filn3
         print ' Location       = ',loca3
         print ' Mission        = ',mis3
         print ' Object         = ',obsd3[0]
         print ' Obs_ID         = ',obsd3[1]
         if mis3 in ['SUZAKU']:
            print ' Energy         = ',ch[3],'eV'
         else:
            print ' Channel        = ',ch[3]
         print ' BG Subtracted  = ',bsub3
         print ' Resolution     = ',str(bsz3)+'s'
         print ' No. of PCUs    = ',pcus3
         print ' Flavour        = ',flv3
         print ' FITSGenie Ver. = ',v3
      print ''
      print 'Other Info:'
      print ' Global Flavour = ',flavour
      print ' Obs. Starttime = ',str(tst1)+'s (0.0s)'
                                                      # The start of the observation
      print ' Obs. Endtime   = ',str(oet+tst1)+'s ('+str(oet)+'s)'
                                                      # The end of the observation
      print ' Data Starttime = ',str(dst+tst1)+'s ('+str(dst)+'s)'
                                                      # The start of the data set (i.e.
                                                      # after GTI considerations and
                                                      # clipping)
      print ' Data Endtime   = ',str(det+tst1)+'s ('+str(det)+'s)'
                                                      # The end of the data set
      print ' Bin-size       = ',str(binning)+'s'
      print ' Background     = ',str(bg)+'cts/s/PCU'
      print ' Folded         = ',folded
      if folded:
         print ' Fold Period    = ',period
      print ' Errorbars      = ',es
      print ' Delineated     = ',ls
      print ' Colour-coded   = ',cs

   #-----'reflav' Option---------------------------------------------------------------

   elif plotopt=='reflav':

      print 'Please give a new flavour.'

      try:
         nflavour=raw_input('Flavour: ')
         assert nflavour!=''
         flavour=nflavour
         if flavour=='':
            qflav=''
         else:
            qflav=' "'+flavour+'"'
         print 'Flavour set to "'+flavour+'"'
      except:
         print 'Invalid flavour!  Flavour remains "'+flavour+'"'


   #-----'help' Option-----------------------------------------------------------------

   elif plotopt in ['help','?']:                      # Display instructions

      print 'Instructions:'
      print ''

      give_inst()                                     # Re-call the instructions list,
                                                      # defined as the get_inst()
                                                      # function in initialisation


   #-----'quit' Option-----------------------------------------------------------------
   
   elif plotopt not in ['quit','exit']:               # Invalid command if none of the
                                                      # if statements triggered and no
                                                      # 'q' given

      print 'Invalid command!'

   if plotopt not in ['quit','exit']:
      print ''
      print ' --------------------'


#-----Exiting Interactive Mode---------------------------------------------------------

print ''
print 'Goodbye!'                                           


#-----Footer---------------------------------------------------------------------------

pan.signoff()

\end{minted}

\section{Spec Angel}

\begin{minted}[fontsize=\scriptsize]{python}
\end{minted}

\section{Back Hydra}

\begin{minted}[fontsize=\scriptsize]{python}
\end{minted}

\section{PAN Lib}

\begin{minted}[fontsize=\scriptsize]{python}
\end{minted}

\section{XTE-PAN Lib}

\begin{minted}[fontsize=\scriptsize]{python}
\end{minted}

\section{Mode Get}

\begin{minted}[fontsize=\scriptsize]{bash}
\end{minted}